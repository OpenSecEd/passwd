A lot of user authentication is based on passwords.
We use password policies to aid users in selecting a secure password.
Unfortunately, research has shown that the common password-polices do not have 
the expected effect: users can still choose easy-to-guess passwords and the 
policies actually makes guessing easier.
It is thus important to \emph{scientifically} evaluate the actual effects of 
any user-authentication mechanism, otherwise our security might be at risk.
Here we will focus on exactly that.
More specifically, after this lab you should be able to
\begin{itemize}
  \item \emph{evaluate} the effective security by considering security and 
    usability.
  \item \emph{analyse} research results in usable security and \emph{apply} 
    those relevant to a given situation.
  \item \emph{design} security policies aligned with usability.
\end{itemize}

To do this, we must be familiar with several topics: usability, cryptography, 
information theory and the scientific method.
Usability is covered in Chapter~2 of 
\citetitle{Anderson2008sea}~\cite{Anderson2008sea}.
The cryptography is covered in Chapter~5 of the same 
book~\cite{Anderson2008sea} and is complemented by the lecture 
\citetitle{BoskHighLevelCrypto}~\cite{BoskHighLevelCrypto}.
The information theory is best covered in 
\citetitle{Ueltschi2013se}~\cite{Ueltschi2013se}.
Finally, the scientific method in general is discussed in 
\citetitle{ComputerSecurityExperiments}~\cite{ComputerSecurityExperiments}.
This is complemented by
\citetitle{GuessAgainAndAgain}~\cite{GuessAgainAndAgain},
\citetitle{OfPasswordsAndPeople}~\cite{OfPasswordsAndPeople} and
\citetitle{CanLongPasswordsBeSecureAndUsable}~\cite{CanLongPasswordsBeSecureAndUsable}.
