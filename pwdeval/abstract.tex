A lot of user authentication is based on passwords.
We use password policies to aid users in selecting a secure password.
Unfortunately, research has shown that the common password-polices do not have 
the expected effect: users can still choose easy-to-guess passwords and the 
policies actually makes guessing easier.
It is thus important to \emph{scientifically} evaluate the actual effects of 
any user-authentication mechanism, otherwise our security might be at risk.
Here we will focus on exactly that.
More specifically, after this lab you should be able to
\begin{itemize}
  \item \emph{evaluate} the effective security by considering security and 
    usability.
  \item \emph{analyse} research results in usable security and \emph{apply} 
    those relevant to a given situation.
  \item \emph{design} security policies aligned with usability.
\end{itemize}

To do this, we must be familiar with several topics: 
usability~\cite[Ch.~2]{Anderson2008sea}, 
cryptography~\cite[Ch.~5]{Anderson2008sea}~\cite{BoskHighLevelCrypto},  
information theory~\cite{Ueltschi2013se} and the scientific 
method~\cite{ComputerSecurityExperiments}.
The main contents is some research papers on password security and usability:
\citetitle{GuessAgainAndAgain}~\cite{GuessAgainAndAgain},
\citetitle{OfPasswordsAndPeople}~\cite{OfPasswordsAndPeople}, 
\citetitle{CanLongPasswordsBeSecureAndUsable}~\cite{CanLongPasswordsBeSecureAndUsable} 
and
\citetitle{PasswordLifeCycle}~\cite{PasswordLifeCycle};
complemented by a paper on the usability of password managers:
\citetitle{UsabilityEvaluationOfPasswordManagers}~\cite{UsabilityEvaluationOfPasswordManagers}.
