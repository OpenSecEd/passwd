\title{Seminar: Evaluating and designing authentication}

\author{%
  Daniel Bosk
  \and
  Lennart Franked
}
\institute{%
  Department of Information Systems and Technology\\
  Mid Sweden University, SE-851\,70 Sundsvall
}

\maketitle

\begin{abstract}
  The part of security that perhaps most affect the users is user authentication.
The predominant mechanism to achieve this is passwords.
Thus, design decisions in this are are important for both the usability and the 
security of the system.

During this seminar you will train your ability to comprehend and apply 
research results in the area of security and usable security.
You will combine results from different areas to analyse different aspects and 
to evaluate the security and usability of different designs.

We need Chap.~2 \enquote{Usability and Psychology} of \cite{Anderson2008sea}.
Further, we need a basic understanding of information 
theory~\cite{Shannon1948amt}, for this you are recommended to read 
\citetitle{Ueltschi2013se}~\cite{Ueltschi2013se}.
Finally, we will discuss the results of 
\citetitle{Komanduri2011opa}~\cite{Komanduri2011opa} and 
\citetitle{Komanduri2014can}~\cite{Komanduri2014can}.

\end{abstract}


\section{Introduction}%
\label{sec:intro}

User authentication is present in most systems.
There is one security mechanism which can be found almost everywhere, which is 
intended to solve this problem: passwords.
From a usability perspective, passwords generally perform very poorly.
This will, of course, also yield security implications.

Usually, in systems using passwords, the password selection of the users are 
governed by some password-composition policy to help (or force) the users to 
select strong passwords.
Thus, how these policies are designed has great impact on the resulting 
passwords the users choose.
This impact is not always what is expected, in fact, sometimes a password 
policy can result in weaker passwords.

There is no indication that passwords will be replaced any time soon, so if we 
must use passwords, we would better use them well.
This is the goal of this assignment.


\section{Assignment}%
\label{sec:tasks}

The seminar is divided into two seminar sessions.
In the first (\cref{password-policies}), we will evaluate how 
password-composition policies affects users.
In the second (\cref{password-managers-tools}), we will evaluate how users cope 
with passwords and evaluate the usability of password managers.

You will read research papers for these sessions.
While reading, write down your thoughts.
Let these questions guide your reading:
\begin{itemize}
  \item What are the main results of the research paper?
  \item How did they conclude them?
    \Ie what is the research method?
\end{itemize}

You are encouraged to discuss your thoughts in the working groups before the 
seminar.
The groups at the seminars will be randomly chosen to maximize exchanges of 
ideas.

\subsection{Password-composition policies}%
\label{password-policies}

\paragraph{Before the seminar session}

First you must read the chapter \enquote{Usability and 
  Psychology}~\cite[Ch.~2]{Anderson2008sea}.
Further, you need a basic understanding of information theory, for this you are 
recommended to read \citetitle{Ueltschi2013se}~\cite{Ueltschi2013se}.

Read the papers
\citetitle{OfPasswordsAndPeople}~\cite{OfPasswordsAndPeople}, 
\citetitle{GuessAgainAndAgain}~\cite{GuessAgainAndAgain},
\citetitle{CanLongPasswordsBeSecureAndUsable}~\cite{CanLongPasswordsBeSecureAndUsable} 
and, finally,
\citetitle{PasswordLifeCycle}~\cite{PasswordLifeCycle}.
In these papers the authors studied how password-composition policies affects 
the interplay between password security and usability and how users cope with 
passwords.

After reading and reflection, think about the following questions:
\begin{itemize}
  \item How do these results compare to your experience of what is used in 
    practice?
  \item What is your strategy for remembering passwords?
  \item How do you react to different password policies?
  \item What would be a good password-composition policy?
    Why would that be good?
  \item How much information do you think a password policy reveals about the 
    passwords?
    Is there any way we can estimate that?
  \item Is it fine to write down passwords or not?
  \item In the situation where you have forgotten you password,
  	what kind of password recovery schemes have you encountered?
  \item What is your reaction towards the different password recovery schemes?  
  \item How should password recovery be dealt with in a secure way?
  \item What problems do you perceive with authentication of users in general?
  \item In which situations are passwords suitable and in which are they not?
\end{itemize}
Finally, look at the University's password-composition policy, what are the 
strengths and weaknesses of this policy?
Have you encountered any service which has a good password policy?

\paragraph{During the seminar session}

During the seminar we will first discuss the papers and your reflections from 
reading them.
We will first do this in groups of 3--4 students.
(The groups will have 30 minutes for discussion.)
Then each group will summarize their discussions for the whole class.
(We will spend 15 minutes for all groups to present.)

After a short break, each group will design a password-composition policy for
\begin{itemize}
  \item BankID and Mobile BankID,
  \item a web-mail account,
  \item a company login account,
  \item an encrypted hard-drive;
\end{itemize}
drawing from both the papers and the discussions.
(We allocate 25 minutes for this.)
Finally, every group will present their policy and analysis, then we will 
evaluate it together.
(We spend the last 20 minutes on this.)

\subsection{Password managers and other tools}%
\label{password-managers-tools}

\paragraph{Before the seminar session}

Read the paper 
\citetitle{UsabilityEvaluationOfPasswordManagers}~\cite{UsabilityEvaluationOfPasswordManagers}.
In this paper the authors examines the usability of three password managers and 
compares them.

You should then evaluate a password manager (you choose which), \eg:
\begin{itemize}
  \item KeePass\{X,XC\},
  \item LastPass,
  \item 1Password,
  \item LessPass,
  \item BitWarden,
  \item PassBolt,
  \item Password Safe;
\end{itemize}
and the following three authentication mechanisms (you must evaluate all):
\begin{itemize}
  \item (Mobile) BankID\footnote{%
      URL: \url{https://www.bankid.com}.
    },
  \item WebAuthn\footnote{%
      URL: \url{https://webauthn.io}.
      Note that you might have to use a smartphone as those have more hardware 
      support than most laptops.
    } and
  \item Identity Mixer\footnote{%
      URL: \url{https://idemixdemo.mybluemix.net/}.
    }.
\end{itemize}

After this, think of a few services, tools or devices that you frequently use 
and where you must authenticate.
Think about the authentication in those situations: are they properly designed, 
how would you like to change them and why?

\paragraph{During the seminar session}

During the second part of the seminar we will discuss your experiences, 
thoughts and suggestions for improvements in groups.

First, we will discuss (in groups) your experience of the password managers and 
how those relate to the results of the 
paper~\cite{UsabilityEvaluationOfPasswordManagers}.
How did you \enquote{evaluate} the password manager?
What were your conclusions?
Second, we will discuss the advantages and disadvantages of the other 
authentication mechanisms that you tried.
How did you \enquote{evaluate} them, what did you do, how was that experience?
(The groups will have 30 minutes for discussions.)
Then each group will summarize the most important points in class.
(We allocate 15 minutes for this.)

Third, after a short break, we will work in groups and each group will focus on 
a service that they would like to improve.
(The groups have 25 minutes to work.)
After the group is done, it will present its results to the others.
(We spend the last 20 minutes on these presentations.)


\section{Examination}%
\label{sec:exam}

To pass this assignment you need to come prepared and actively participate 
throughout the activities.


\printbibliography
