\title{Seminar: Evaluating and designing authentication}

\author{%
  Daniel Bosk
  \and
  Lennart Franked
}
\institute{%
  Department of Information Systems and Technology\\
  Mid Sweden University, SE-851\,70 Sundsvall
}

\maketitle

\begin{abstract}
  The part of security that perhaps most affect the users is user authentication.
The predominant mechanism to achieve this is passwords.
Thus, design decisions in this are are important for both the usability and the 
security of the system.

During this seminar you will train your ability to comprehend and apply 
research results in the area of security and usable security.
You will combine results from different areas to analyse different aspects and 
to evaluate the security and usability of different designs.

We need Chap.~2 \enquote{Usability and Psychology} of \cite{Anderson2008sea}.
Further, we need a basic understanding of information 
theory~\cite{Shannon1948amt}, for this you are recommended to read 
\citetitle{Ueltschi2013se}~\cite{Ueltschi2013se}.
Finally, we will discuss the results of 
\citetitle{Komanduri2011opa}~\cite{Komanduri2011opa} and 
\citetitle{Komanduri2014can}~\cite{Komanduri2014can}.

\end{abstract}


\section{Introduction}%
\label{sec:intro}

User authentication is present in most systems.
There is one security mechanism which can be found almost everywhere, which is 
intended to solve this problem: passwords.
From a usability perspective, passwords generally perform very poorly.
This will, of course, also yield security implications.

Usually, in systems using passwords, the password selection of the users are 
governed by some password-composition policy to help (or force) the users to 
select strong passwords.
Thus, how these policies are designed has great impact on the resulting 
passwords the users choose.
This impact is not always what is expected, in fact, sometimes a password 
policy can result in weaker passwords.

There is no indication that passwords will be replaced any time soon, so if we 
must use passwords, we would better use them well.
This is the goal of this assignment.


\section{Assignment}%
\label{sec:tasks}

The seminar is divided into several parts.
In the first we will evaluate how password-composition policies affects users.
In the second part, we will evaluate how users cope with passwords and evaluate 
the usability of password managers.

You will read research papers throughout these parts.
While reading, write down your thoughts.
Let these questions guide your reading:
\begin{itemize}
  \item What are the main results of the research paper?
  \item How did they conclude them?
    \Ie what is the research method?
\end{itemize}

\subsection{Password-composition policies}

First you must read the chapter \enquote{Usability and 
  Psychology}~\cite[Ch.~2]{Anderson2008sea}.
Further, you need a basic understanding of information theory, for this you are 
recommended to read \citetitle{Ueltschi2013se}~\cite{Ueltschi2013se}.

Read the papers
\citetitle{GuessAgainAndAgain}~\cite{GuessAgainAndAgain},
\citetitle{OfPasswordsAndPeople}~\cite{OfPasswordsAndPeople}, 
\citetitle{CanLongPasswordsBeSecureAndUsable}~\cite{CanLongPasswordsBeSecureAndUsable} 
and
\citetitle{PasswordLifeCycle}~\cite{PasswordLifeCycle}.
In these papers the authors studied how password-composition policies affects 
the interplay between password security and usability and how users cope with 
passwords.

After reading and reflection, think about the following questions:
\begin{itemize}
  \item How do these results compare to your experience of what is used in 
    practice?
  \item What is your strategy for remembering passwords?
  \item How do you react to different password policies?
  \item What would be a good password-composition policy?
    Why would that be good?
  \item How much information do you think a password policy reveals about the 
    passwords?
    Is there any way we can estimate that?
  \item Is it fine to write down passwords or not?
  \item In the situation where you have forgotten you password,
  	what kind of password recovery schemes have you encountered?
  \item What is your reaction towards the different password recovery schemes?  
  \item How should password recovery be dealt with in a secure way?
  \item What problems do you perceive with authentication of users in general?
  \item In which situations are passwords suitable and in which are they not?
\end{itemize}
Finally, look at the University's password-composition policy, what are the 
strengths and weaknesses of this policy?

During the seminar we will first discuss the papers and your reflections from 
reading them.
After that we will work in groups of 3--4 students.
We will first discuss how we can estimate how much a password policy reveals 
about the passwords, \ie how the password policy can make guessing easier.
Then every group will design a password-composition policy, drawing from both 
the papers and the discussions.
Finally, every group will present their policy and analysis, then we will 
evaluate it together.

\subsection{Password managers and other tools}

Read the paper 
\citetitle{UsabilityEvaluationOfPasswordManagers}~\cite{UsabilityEvaluationOfPasswordManagers}.
In this paper the authors examines the usability of three password managers and 
compares them.

You should then look a password manager (you choose which) and at the following 
three authentication mechanisms:
Mobile BankID\footnote{%
  URL: \url{https://www.bankid.com}.
},
WebAuthn\footnote{%
  URL: \url{https://webauthn.io}.
  Note that you might have to use a smartphone as those have more hardware 
  support than most laptops.
} and
Identity Mixer\footnote{%
  URL: \url{https://idemixdemo.mybluemix.net/}.
}.

After this, think of a few services, tools or devices that you frequently use 
and where you must authenticate.
Think about the authentication in those situations: are they properly designed, 
how would you like to change them and why?

During the second part of the seminar we will discuss your experiences, 
thoughts and suggestions for improvements.
First we will discuss your experience of the password managers and how those 
relate to the results of the 
paper~\cite{UsabilityEvaluationOfPasswordManagers}.
How did you \enquote{evaluate} the password manager?

Second, we will discuss the advantages and disadvantages of the other 
authentication mechanisms that you tried.
How did you \enquote{evaluate}, what did you do, how was that experience?

Third, we will work in groups and each group will focus on a service that they 
would like to improve.
After the group is done, it will present its results to the others.


\section{Examination}%
\label{sec:exam}

To pass this assignment you need to come prepared and actively participate 
throughout the activities.


\printbibliography
