\title{Seminar: Evaluating and designing authentication}

\author{%
  Daniel Bosk
  \and
  Lennart Franked
}
\institute{%
  Department of Information Systems and Technology\\
  Mid Sweden University, SE-851\,70 Sundsvall
}

\maketitle

\begin{abstract}
  The part of security that perhaps most affect the users is user authentication.
The predominant mechanism to achieve this is passwords.
Thus, design decisions in this are are important for both the usability and the 
security of the system.

During this seminar you will train your ability to comprehend and apply 
research results in the area of security and usable security.
You will combine results from different areas to analyse different aspects and 
to evaluate the security and usability of different designs.

We need Chap.~2 \enquote{Usability and Psychology} of \cite{Anderson2008sea}.
Further, we need a basic understanding of information 
theory~\cite{Shannon1948amt}, for this you are recommended to read 
\citetitle{Ueltschi2013se}~\cite{Ueltschi2013se}.
Finally, we will discuss the results of 
\citetitle{Komanduri2011opa}~\cite{Komanduri2011opa} and 
\citetitle{Komanduri2014can}~\cite{Komanduri2014can}.

\end{abstract}


\section{Introduction}%
\label{sec:intro}

User authentication is present in most systems.
There is one security mechanism which can be found almost everywhere, which is 
intended to solve this problem: passwords.
From a usability perspective, passwords generally perform very poorly.
This will, of course, also yield security implications.

Usually, in systems using passwords, the password selection of the users are 
governed by some password-composition policy to help (or force) the users to 
select strong passwords.
Thus, how these policies are designed has great impact on the resulting 
passwords the users choose.
This impact is not always what is expected, in fact, sometimes a password 
policy can result in weaker passwords.

There is no indication that passwords will be replaced any time soon, so if we 
must use passwords, we would better use them well.
This is the goal of this assignment.


\section{Assignment}%
\label{sec:tasks}

The seminar is divided into two seminar sessions.
In the first (\cref{password-policies}), we will evaluate how 
password-composition policies, both their security and their usability.
In the second (\cref{password-managers-tools}), we will evaluate how users cope 
with passwords and evaluate the usability of password managers.

You will read research papers for these sessions.
While reading, write down your thoughts.
Let these questions guide your reading:
\begin{itemize}
  \item What are the main results of the research paper?
  \item How did they conclude them?
    \Ie what is the research method?
\end{itemize}

Discuss your thoughts in the working groups before each seminar.
The groups at the seminars will be randomly chosen to maximize exchanges of 
ideas.

\subsection{Password-composition policies}%
\label{password-policies}

\paragraph{Before the seminar session}

First you must read the chapter \enquote{Usability and 
  Psychology}~\cite[Ch.~2]{Anderson2008sea}.
Further, you need a basic understanding of information theory, for this you are 
recommended to read \citetitle{Ueltschi2013se}~\cite{Ueltschi2013se}.

Start by estimating the strength of the following two password-composition 
policies:
\begin{itemize}
  \item At least eight characters chosen uniformly randomly.
    The characters include lower and upper case, digits and special characters.

  \item Four words chosen uniformly randomly.
    Say the words are Swedish, then there are approximately 125\,000 words in 
    SAOL (the common dictionary of Swedish).

  \item Add to the first policy the requirement of at least one character from 
    each character group.
\end{itemize}
How strong are they?
Order them according to their strengths.

Read the papers
\citetitle{OfPasswordsAndPeople}~\cite{OfPasswordsAndPeople}, 
\citetitle{GuessAgainAndAgain}~\cite{GuessAgainAndAgain},
\citetitle{CanLongPasswordsBeSecureAndUsable}~\cite{CanLongPasswordsBeSecureAndUsable} 
and, finally,
\citetitle{PasswordLifeCycle}~\cite{PasswordLifeCycle}.
In these papers the authors studied how password-composition policies affects 
the interplay between password security and usability and how users cope with 
passwords.
Focus on these questions:
\begin{itemize}
  \item What are the main results of the research paper?
  \item How did they conclude them?
    \Ie what is the research method?
\end{itemize}

After reading, reflect on what you have read, think about the following 
questions:
\begin{itemize}
  \item How do these results compare to your experience of what is used in 
    practice?
  \item How are your password strategies compared to those in the papers?
  \item What would be a good password policy?
    Why would that be good?
  \item Is it fine to write down passwords or not?
    It depends?
  \item How to recover from failed states; \eg forgotten passwords, lost keys?
  \item What problems do you perceive with authentication of users in general?
  \item In which situations are passwords suitable and in which are they not?
\end{itemize}

Crack the following three passwords available in the two files:
\begin{center}
  \url{https://github.com/OpenSecEd/passwd/releases/download/v1.1/unix-passwd.txt}
\end{center}
and
\begin{center}
  \url{https://github.com/OpenSecEd/passwd/releases/download/v1.1/win-pwd.txt}
\end{center}
There are instructions for how to get going with this in \cref{cracking}.
(You should work in your groups and make this as fast as possible.)

Discuss the papers and your reflections in the group, how hard are passwords to 
crack?
For the papers, try to answer: what questions does each paper try to answer, 
how do they do this?

\paragraph{During the seminar session}

During the seminar we will first let the groups summarize their discussions 
from before the seminar: the papers, the reflections and experience from 
password cracking.
(We will take 30 minutes for these summaries.)

After a short break, each group will design a password policy for one of the 
following:
\begin{itemize}
  \item BankID and Mobile BankID,
  \item a web-mail account,
  \item a company login account,
  \item an encrypted hard-drive;
\end{itemize}
drawing from both the papers and the discussions.
(We allocate 20 minutes for this.)
Finally, every group will present their policy and analysis, then we will 
evaluate it together.
(We spend the last 20 minutes on this.)

\subsection{Password managers and other tools}%
\label{password-managers-tools}

\paragraph{Before the seminar session}

Read the paper 
\citetitle{UsabilityEvaluationOfPasswordManagers}~\cite{UsabilityEvaluationOfPasswordManagers}.
In this paper the authors examines the usability of three password managers and 
compares them.
Discuss the paper in the group: what questions does the paper answer, how do 
they do this?

The group should then evaluate one password manager.
Discuss in the group what interesting factors to look at.
There are numerous password managers, some of the most popular are \eg:
\begin{itemize}
  \item KeePass\{X,XC\},
  \item LastPass,
  \item 1Password,
  \item LessPass,
  \item BitWarden,
  \item PassBolt,
  \item Password Safe.
\end{itemize}
(Remember, the group only needs to evaluate one.)

Then the group should also evaluate the following authentication mechanisms:
\begin{itemize}
  \item (Mobile) BankID\footnote{%
      URL: \url{https://www.bankid.com}.
    },
  \item WebAuthn\footnote{%
      URL: \url{https://webauthn.io}.
      Note that you might have to use a smartphone as those have more hardware 
      support than most laptops.
    } and
  \item Identity Mixer\footnote{%
      URL: \url{https://idemixdemo.mybluemix.net/}.
    }.
\end{itemize}

After this, think of a few services, tools or devices that you frequently use 
and where you must authenticate.
Think about the authentication in those situations: are they properly designed, 
how would you like to change them and why?

\paragraph{During the seminar session}

During the second part of the seminar we will discuss your experiences, 
thoughts and suggestions for improvements.
We will do this in groups (randomly chosen to spread the ideas).

First, we will discuss (in groups) your experience of the password managers and 
how those relate to the results of the 
paper~\cite{UsabilityEvaluationOfPasswordManagers}.
How did you \enquote{evaluate} the password manager?
What were your conclusions?
(The groups will have 30 minutes to discuss this.)

We will summarize the discussions and, particularly, differences between groups 
in full class.
(We will spend 15 minutes on this.)

After a short break, we will discuss the advantages and disadvantages of the 
other authentication mechanisms that you tried.
How did you \enquote{evaluate} them, what did you do, how was that experience?
What service did you want to improve, how and why?
(The groups will have 30 minutes for discussions.)
Then each group will summarize the most important parts of their discussion in 
class: any changes of mind, diverging opinions?
(We allocate 15 minutes for this.)


\section{Examination}%
\label{sec:exam}

To pass this assignment you need to come prepared and actively participate 
throughout the activities.


\printbibliography


\appendix
\section{How to crack passwords}%
\label{cracking}

\paragraph{Cracking programs}

The papers above used some password-cracking software.
In addition, on the website
\begin{center}
  \url{http://sectools.org/tag/crackers/}
\end{center}
you can find a list of programs for password cracking.
You are free to use any program to solve this, however, there is a Docker image 
available to make things easy.

The Windows hash is an old NTLM hash, which means that it is not 
salted\footnote{%
  Consider this when choosing your method for cracking.
}.
The UNIX hash is salted and uses Blowfish (OpenBSD).

\paragraph{How to obtain the password hashes}

For a UNIX-like operating system the password hashes with corresponding salts 
are stored in the file \enquote{/etc/master.passwd} on BSD-based systems such 
as OpenBSD and FreeBSD\@.
In the case of Linux-based systems such as Ubuntu, the file used is 
\enquote{/etc/shadow}.
You need privileges (root) to read this file.

The hashes on a Windows system can be acquired by the program fgdump.
This is available from URL
\begin{center}
  \url{http://www.foofus.net/~fizzgig/fgdump/}.
\end{center}

The hashes in this assignment are already extracted from these files for your 
convenience.
You are going to find the passwords for both Windows and UNIX-like systems.
Thus, \emph{you do not have to use any program like fgdump or unshadow(8) to 
extract them}.

\subsection{A Docker image}%
\label{Dockerfile}

If you have Docker on your computer you can use the container provided here.
It has John the Ripper and Ophcrack preinstalled.
To use it, create a directory on your computer where you have all the files you 
would like to have available.
Then start the container from that directory.
\begin{verbatim}
docker run -it -v $(pwd):/pwdeval dbosk/pwdeval
\end{verbatim}
This will map the current directory to the working directory (\verb'/pwdeval') 
inside the container.
The first time you run it, it will automatically download the image from Docker 
Hub.


\subsection{Instructions for Ophcrack and John the Ripper}

\paragraph{Ophcrack}

The ophcrack(1) program uses a technique called rainbow tables.
What this means is that all password and hash-value combinations are 
precomputed and stored in a huge table.
This is called a hash table.
The rainbow table is a special case of hash table, the benefit is that it is 
smaller than a conventional hash table.
The hash table reduces the problem of cracking the password to searching this 
huge table.

The alternative approach is to compute the hash value for each guess, this 
takes time and this time is what is saved by using a hash table.
However, this comes with some compromises, the hash tables (and even rainbow 
tables) requires a lot of computational resources to produce.
They also requires great resources to use, they must preferably fit in the 
computers primary memory.
Hence the use of hash tables and rainbow tables is a trade-off between 
computational and storage resources.

Because of the space limitations of this method it can easily be countered by 
adding a salt to the hash.
This means that the rainbow table must increase too much in size to be 
feasible.
Unfortunately, some Windows hashes are not salted, so this method can be used 
on those hashes (at least in some cases).
UNIX-like systems has a longer tradition of using salts, so this method is not 
feasible on those hashes.

You will find ophcrack(1) in the package manager of most UNIX-like systems.
You can also find it on URL
\begin{center}
  \url{http://ophcrack.sourceforge.net/}.
\end{center}
You also need a few rainbow tables to be able to use the program.
You can find these on the website above.
Choose your tables carefully.

\paragraph{John the Ripper}

John the Ripper is a terminal-based program using many different ways of 
cracking passwords.
It has the possibility of brute-force attacks, dictionary attacks, and the 
possibility of using rules to modify the words in the dictionary (\eg 
\enquote{leet-speak}).
Naturally, these methods takes much longer time to use than a rainbow table, 
since all computations are done in real-time.

The program can be found in the package manager of most UNIX-like systems, or 
on URL
\begin{center}
  \url{http://www.openwall.com/john/}.
\end{center}
You are recommended to use the \enquote{Community Enhanced Version}.

To have a short summary of the possible arguments to pass to John the Ripper, 
just run the command \enquote{john} in the terminal without any arguments.
See \cref{lst:john}.
You can also read the manual page john(1).

\begin{lstlisting}[float,caption={Output from John the Ripper in the 
terminal.},label={lst:john},breaklines=false]
$ john
John the Ripper password cracker, version 1.7.8
Copyright (c) 1996-2011 by Solar Designer
Homepage: http://www.openwall.com/john/

Usage: john [OPTIONS] [PASSWORD-FILES]
--single                   "single crack" mode
--wordlist=FILE --stdin    wordlist mode, read words from FILE or stdin
--rules                    enable word mangling rules for wordlist mode
--incremental[=MODE]       "incremental" mode [using section MODE]
--external=MODE            external mode or word filter
--stdout[=LENGTH]          just output candidate passwords [cut at LENGTH]
--restore[=NAME]           restore an interrupted session [called NAME]
--session=NAME             give a new session the NAME
--status[=NAME]            print status of a session [called NAME]
--make-charset=FILE        make a charset, FILE will be overwritten
--show                     show cracked passwords
--test[=TIME]              run tests and benchmarks for TIME seconds each
--users=[-]LOGIN|UID[,..]  [do not] load this (these) user(s) only
--groups=[-]GID[,..]       load users [not] of this (these) group(s) only
--shells=[-]SHELL[,..]     load users with[out] this (these) shell(s) only
--salts=[-]COUNT           load salts with[out] at least COUNT passwords only
--format=NAME              force hash type NAME: DES/BSDI/MD5/BF/AFS/LM/crypt
--save-memory=LEVEL        enable memory saving, at LEVEL 1..3
$
\end{lstlisting}

The most interesting arguments are
\begin{itemize}
  \item --show,
  \item --wordlist,
  \item --rules, and
  \item --incremental=all.
\end{itemize}
Note that \enquote{--wordlist} and \enquote{--stdin} are separate arguments.
The first reads words from a file (provided a filename) while the latter reads 
words from standard input.
You can read more about this in the manual by typing \enquote{man 1 john} in 
the terminal.

You will find links to different wordlists to use in the following URL\@:
\begin{center}
  \url{http://sectools.org/tag/crackers/}.
\end{center}
Choose your wordlists with care.
You also have the script \enquote{pwdstream.py} (\cref{pwdstream}) to help 
generate a stream of passwords, see \enquote{./pwdstream.py -h} for details.


\subsection{Password guess generator}%
\label{pwdstream}

For this lab there is also a password guess generator.
This can be used to better control what guesses are used while cracking.
It will output a stream of passwords, one per line, on standard out, hence you 
can pipe this to John the Ripper using the \enquote{--stdin} option.

You can find its source code downloadable from the URL
\begin{center}
\url{https://github.com/OpenSecEd/passwd/releases/download/v1.1/pwdstream.py}.
\end{center}

%\lstinputlisting{pwdstream.py}


\subsection*{Acknowledgement}

This work was originally based on previous work by Rahim Rahmani and Curt-Olof 
Klasson.
It has evolved much since then, essentially only the Windows password hash is 
the same.

This work is released under the Creative Commons Attribution-ShareAlike 3.0 
Unported license.
To view a copy of this license, visit 
\url{http://creativecommons.org/licenses/by-sa/3.0/}.
You can find the original source code in URL 
\url{https://github.com/OpenSecEd/passwd/pwdguess/}.




