\title{Seminar: Evaluating and designing authentication}

\author{%
  Daniel Bosk
  \and
  Lennart Franked
}
\institute{%
  Department of Information Systems and Technology\\
  Mid Sweden University, SE-851\,70 Sundsvall
}

\maketitle

\begin{abstract}
  The part of security that perhaps most affect the users is user authentication.
The predominant mechanism to achieve this is passwords.
Thus, design decisions in this are are important for both the usability and the 
security of the system.

During this seminar you will train your ability to comprehend and apply 
research results in the area of security and usable security.
You will combine results from different areas to analyse different aspects and 
to evaluate the security and usability of different designs.

We need Chap.~2 \enquote{Usability and Psychology} of \cite{Anderson2008sea}.
Further, we need a basic understanding of information 
theory~\cite{Shannon1948amt}, for this you are recommended to read 
\citetitle{Ueltschi2013se}~\cite{Ueltschi2013se}.
Finally, we will discuss the results of 
\citetitle{Komanduri2011opa}~\cite{Komanduri2011opa} and 
\citetitle{Komanduri2014can}~\cite{Komanduri2014can}.

\end{abstract}


\section{Introduction}%
\label{sec:intro}

User authentication is present in most systems.
There is one security mechanism which can be found almost everywhere, which is 
intended to solve this problem: passwords.
From a usability perspective, passwords generally perform very poorly.
This will, of course, also yield security implications.

Usually, in systems using passwords, the password selection of the users are 
governed by some password-composition policy to help (or force) the users to 
select strong passwords.
Thus, how these policies are designed has great impact on the resulting 
passwords the users choose.
This impact is not always what is expected, in fact, sometimes a password 
policy can result in weaker passwords.

There is no indication that passwords will be replaced any time soon, so if we 
must use passwords, we would better use them well.
This is the goal of this assignment.


\section{Assignment}%
\label{sec:tasks}

The seminar is divided into two seminar sessions.
In the first (\cref{password-policies}), we will evaluate how 
password-composition policies affects users.
In the second (\cref{password-managers-tools}), we will evaluate how users cope 
with passwords and evaluate the usability of password managers.

You will read research papers for these sessions.
While reading, write down your thoughts.
Let these questions guide your reading:
\begin{itemize}
  \item What are the main results of the research paper?
  \item How did they conclude them?
    \Ie what is the research method?
\end{itemize}

Discuss your thoughts in the working groups before each seminar.
The groups at the seminars will be randomly chosen to maximize exchanges of 
ideas.

\subsection{Password-composition policies}%
\label{password-policies}

\paragraph{Before the seminar session}

First you must read the chapter \enquote{Usability and 
  Psychology}~\cite[Ch.~2]{Anderson2008sea}.
Further, you need a basic understanding of information theory, for this you are 
recommended to read \citetitle{Ueltschi2013se}~\cite{Ueltschi2013se}.

Read the papers
\citetitle{OfPasswordsAndPeople}~\cite{OfPasswordsAndPeople}, 
\citetitle{GuessAgainAndAgain}~\cite{GuessAgainAndAgain},
\citetitle{CanLongPasswordsBeSecureAndUsable}~\cite{CanLongPasswordsBeSecureAndUsable} 
and, finally,
\citetitle{PasswordLifeCycle}~\cite{PasswordLifeCycle}.
In these papers the authors studied how password-composition policies affects 
the interplay between password security and usability and how users cope with 
passwords.
Focus on these questions:
\begin{itemize}
  \item What are the main results of the research paper?
  \item How did they conclude them?
    \Ie what is the research method?
\end{itemize}

After reading, reflect on what you have read, think about the following 
questions:
\begin{itemize}
  \item How do these results compare to your experience of what is used in 
    practice?
  \item How are your password strategies compared to those in the papers?
  \item What would be a good password policy?
    Why would that be good?
  \item Is it fine to write down passwords or not?
    It depends?
  \item How to recover from failed states; \eg forgotten passwords, lost keys?
  \item What problems do you perceive with authentication of users in general?
  \item In which situations are passwords suitable and in which are they not?
\end{itemize}

Discuss the papers and your reflections in the group.
For the papers, try to answer: what questions does each paper try to answer, 
how do they do this?

\paragraph{During the seminar session}

During the seminar we will first let the groups summarize their discussions 
from before the seminar: the papers and the reflections.
(We will take 30 minutes for these summaries.)

After a short break, each group will design a password policy for one of the 
following:
\begin{itemize}
  \item BankID and Mobile BankID,
  \item a web-mail account,
  \item a company login account,
  \item an encrypted hard-drive;
\end{itemize}
drawing from both the papers and the discussions.
(We allocate 20 minutes for this.)
Finally, every group will present their policy and analysis, then we will 
evaluate it together.
(We spend the last 20 minutes on this.)

\subsection{Password managers and other tools}%
\label{password-managers-tools}

\paragraph{Before the seminar session}

Read the paper 
\citetitle{UsabilityEvaluationOfPasswordManagers}~\cite{UsabilityEvaluationOfPasswordManagers}.
In this paper the authors examines the usability of three password managers and 
compares them.
Discuss the paper in the group: what questions does the paper answer, how do 
they do this?

The group should then evaluate one password manager.
Discuss in the group what interesting factors to look at.
There are numerous password managers, some of the most popular are \eg:
\begin{itemize}
  \item KeePass\{X,XC\},
  \item LastPass,
  \item 1Password,
  \item LessPass,
  \item BitWarden,
  \item PassBolt,
  \item Password Safe.
\end{itemize}
(Remember, the group only needs to evaluate one.)

Then the group should also evaluate the following authentication mechanisms:
\begin{itemize}
  \item (Mobile) BankID\footnote{%
      URL: \url{https://www.bankid.com}.
    },
  \item WebAuthn\footnote{%
      URL: \url{https://webauthn.io}.
      Note that you might have to use a smartphone as those have more hardware 
      support than most laptops.
    } and
  \item Identity Mixer\footnote{%
      URL: \url{https://idemixdemo.mybluemix.net/}.
    }.
\end{itemize}

After this, think of a few services, tools or devices that you frequently use 
and where you must authenticate.
Think about the authentication in those situations: are they properly designed, 
how would you like to change them and why?

\paragraph{During the seminar session}

During the second part of the seminar we will discuss your experiences, 
thoughts and suggestions for improvements.
We will do this in groups (randomly chosen to spread the ideas).

First, we will discuss (in groups) your experience of the password managers and 
how those relate to the results of the 
paper~\cite{UsabilityEvaluationOfPasswordManagers}.
How did you \enquote{evaluate} the password manager?
What were your conclusions?
(The groups will have 30 minutes to discuss this.)

We will summarize the discussions and, particularly, differences between groups 
in full class.
(We will spend 15 minutes on this.)

After a short break, we will discuss the advantages and disadvantages of the 
other authentication mechanisms that you tried.
How did you \enquote{evaluate} them, what did you do, how was that experience?
What service did you want to improve, how and why?
(The groups will have 30 minutes for discussions.)
Then each group will summarize the most important parts of their discussion in 
class: any changes of mind, diverging opinions?
(We allocate 15 minutes for this.)


\section{Examination}%
\label{sec:exam}

To pass this assignment you need to come prepared and actively participate 
throughout the activities.


\printbibliography
