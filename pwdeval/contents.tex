\title{Seminar: Evaluating and designing authentication}

\author{%
  Daniel Bosk
  \and
  Lennart Franked
}
\institute{%
  Department of Information Systems and Technology\\
  Mid Sweden University, SE-851\,70 Sundsvall
}

\maketitle

\begin{abstract}
  A lot of user authentication is based on passwords.
We use password policies to aid users in selecting a secure password.
Unfortunately, research has shown that the common password-polices do not have 
the expected effect: users can still choose easy-to-guess passwords and the 
policies actually makes guessing easier.
It is thus important to \emph{scientifically} evaluate the actual effects of 
any user-authentication mechanism, otherwise our security might be at risk.
Here we will focus on exactly that.
More specifically, after this lab you should be able to
\begin{itemize}
  \item \emph{evaluate} the effective security by considering security and 
    usability.
  \item \emph{analyse} research results in usable security and \emph{apply} 
    those relevant to a given situation.
  \item \emph{design} security policies aligned with usability.
\end{itemize}

To do this, we must be familiar with several topics: 
usability~\cite[Ch.~2]{Anderson2008sea}, 
cryptography~\cite[Ch.~5]{Anderson2008sea}~\cite{BoskHighLevelCrypto},  
information theory~\cite{Ueltschi2013se} and the scientific 
method~\cite{ComputerSecurityExperiments}.
The main contents is some research papers on password security and usability:
\citetitle{GuessAgainAndAgain}~\cite{GuessAgainAndAgain},
\citetitle{OfPasswordsAndPeople}~\cite{OfPasswordsAndPeople}, 
\citetitle{CanLongPasswordsBeSecureAndUsable}~\cite{CanLongPasswordsBeSecureAndUsable} 
and
\citetitle{PasswordLifeCycle}~\cite{PasswordLifeCycle};
complemented by a paper on the usability of password managers:
\citetitle{UsabilityEvaluationOfPasswordManagers}~\cite{UsabilityEvaluationOfPasswordManagers}.

\end{abstract}


\section{Introduction}%
\label{sec:intro}

User authentication is present in most systems.
There is one security mechanism which can be found almost everywhere, which is 
intended to solve this problem: passwords.
From a usability perspective, passwords generally perform very poorly.
This will, of course, also yield security implications.

Usually, in systems using passwords, the password selection of the users are 
governed by some password-composition policy to help (or force) the users to 
select strong passwords.
Thus, how these policies are designed has great impact on the resulting 
passwords the users choose.
This impact is not always what is expected, in fact, sometimes a password 
policy can result in weaker passwords.

There is no indication that passwords will be replaced any time soon, so if we 
must use passwords, we would better use them well.
This is the goal of this assignment.


\section{Assignment}%
\label{sec:tasks}

The seminar is divided into several parts.
In the first we will evaluate how password-composition policies affects users.
In the second part, we will evaluate how users cope with passwords and evaluate 
the usability of password managers.

You will read research papers throughout these parts.
While reading, write down your thoughts.
Let these questions guide your reading:
\begin{itemize}
  \item What are the main results of the research paper?
  \item How did they conclude them?
    \Ie what is the research method?
\end{itemize}

You are encouraged to discuss your thoughts in the working groups before the 
seminar.
The groups at the seminars will be randomly chosen to maximize exchanges of 
ideas.

\subsection{Password-composition policies}

First you must read the chapter \enquote{Usability and 
  Psychology}~\cite[Ch.~2]{Anderson2008sea}.
Further, you need a basic understanding of information theory, for this you are 
recommended to read \citetitle{Ueltschi2013se}~\cite{Ueltschi2013se}.

Read the papers
\citetitle{GuessAgainAndAgain}~\cite{GuessAgainAndAgain},
\citetitle{OfPasswordsAndPeople}~\cite{OfPasswordsAndPeople}, 
\citetitle{CanLongPasswordsBeSecureAndUsable}~\cite{CanLongPasswordsBeSecureAndUsable} 
and
\citetitle{PasswordLifeCycle}~\cite{PasswordLifeCycle}.
In these papers the authors studied how password-composition policies affects 
the interplay between password security and usability and how users cope with 
passwords.

After reading and reflection, think about the following questions:
\begin{itemize}
  \item How do these results compare to your experience of what is used in 
    practice?
  \item What is your strategy for remembering passwords?
  \item How do you react to different password policies?
  \item What would be a good password-composition policy?
    Why would that be good?
  \item How much information do you think a password policy reveals about the 
    passwords?
    Is there any way we can estimate that?
  \item Is it fine to write down passwords or not?
  \item In the situation where you have forgotten you password,
  	what kind of password recovery schemes have you encountered?
  \item What is your reaction towards the different password recovery schemes?  
  \item How should password recovery be dealt with in a secure way?
  \item What problems do you perceive with authentication of users in general?
  \item In which situations are passwords suitable and in which are they not?
\end{itemize}
Finally, look at the University's password-composition policy, what are the 
strengths and weaknesses of this policy?
Have you encountered any service which has a good password policy?

During the seminar we will first discuss the papers and your reflections from 
reading them.
We will first do this in groups of 3--4 students.
Then each group will summarize their discussions for the whole class.
After that, each group will design a password-composition policy for
\begin{itemize}
  \item BankID and Mobile BankID,
  \item a web-mail account,
  \item a company login account,
  \item an encrypted hard-drive;
\end{itemize}
drawing from both the papers and the discussions.
Finally, every group will present their policy and analysis, then we will 
evaluate it together.

\subsection{Password managers and other tools}

Read the paper 
\citetitle{UsabilityEvaluationOfPasswordManagers}~\cite{UsabilityEvaluationOfPasswordManagers}.
In this paper the authors examines the usability of three password managers and 
compares them.

You should then evaluate a password manager (you choose which), \eg:
\begin{itemize}
  \item KeePass\{X,XC\},
  \item LastPass,
  \item 1Password,
  \item LessPass,
  \item BitWarden,
  \item PassBolt,
  \item Password Safe;
\end{itemize}
and the following three authentication mechanisms (you must evaluate all):
\begin{itemize}
  \item (Mobile) BankID\footnote{%
      URL: \url{https://www.bankid.com}.
    },
  \item WebAuthn\footnote{%
      URL: \url{https://webauthn.io}.
      Note that you might have to use a smartphone as those have more hardware 
      support than most laptops.
    } and
  \item Identity Mixer\footnote{%
      URL: \url{https://idemixdemo.mybluemix.net/}.
    }.
\end{itemize}

After this, think of a few services, tools or devices that you frequently use 
and where you must authenticate.
Think about the authentication in those situations: are they properly designed, 
how would you like to change them and why?

During the second part of the seminar we will discuss your experiences, 
thoughts and suggestions for improvements.
First we will discuss your experience of the password managers and how those 
relate to the results of the 
paper~\cite{UsabilityEvaluationOfPasswordManagers}.
How did you \enquote{evaluate} the password manager?

Second, we will discuss the advantages and disadvantages of the other 
authentication mechanisms that you tried.
How did you \enquote{evaluate} them, what did you do, how was that experience?

Third, we will work in groups and each group will focus on a service that they 
would like to improve.
After the group is done, it will present its results to the others.


\section{Examination}%
\label{sec:exam}

To pass this assignment you need to come prepared and actively participate 
throughout the activities.


\printbibliography
