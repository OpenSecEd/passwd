\section{How to crack passwords}%
\label{cracking}

\paragraph{Cracking programs}

The papers above used some password-cracking software.
In addition, on the website
\begin{center}
  \url{http://sectools.org/tag/crackers/}
\end{center}
you can find a list of programs for password cracking.
You are free to use any program to solve this, however, there is a Docker image 
available to make things easy.

The Windows hash is an old NTLM hash, which means that it is not 
salted\footnote{%
  Consider this when choosing your method for cracking.
}.
The UNIX hash is salted and uses Blowfish (OpenBSD).

\paragraph{How to obtain the password hashes}

For a UNIX-like operating system the password hashes with corresponding salts 
are stored in the file \enquote{/etc/master.passwd} on BSD-based systems such 
as OpenBSD and FreeBSD\@.
In the case of Linux-based systems such as Ubuntu, the file used is 
\enquote{/etc/shadow}.
You need privileges (root) to read this file.

The hashes on a Windows system can be acquired by the program fgdump.
This is available from URL
\begin{center}
  \url{http://www.foofus.net/~fizzgig/fgdump/}.
\end{center}

The hashes in this assignment are already extracted from these files for your 
convenience.
You are going to find the passwords for both Windows and UNIX-like systems.
Thus, \emph{you do not have to use any program like fgdump or unshadow(8) to 
extract them}.

\subsection{A Docker image}%
\label{Dockerfile}

If you have Docker on your computer you can use the container provided here.
It has John the Ripper and Ophcrack preinstalled.
To download the image, run
\begin{verbatim}
docker pull dbosk/pwdeval
\end{verbatim}

To use it, create a directory on your computer where you have all the files you 
would like to have available.
Then start the container from that directory.
\begin{verbatim}
docker run -it -v $(pwd):/pwdeval dbosk/pwdeval
\end{verbatim}
This will map the current directory to the working directory (\verb'/pwdeval') 
inside the container.


\subsection{Instructions for Ophcrack and John the Ripper}

\paragraph{Ophcrack}

The ophcrack(1) program uses a technique called rainbow tables.
What this means is that all password and hash-value combinations are 
precomputed and stored in a huge table.
This is called a hash table.
The rainbow table is a special case of hash table, the benefit is that it is 
smaller than a conventional hash table.
The hash table reduces the problem of cracking the password to searching this 
huge table.

The alternative approach is to compute the hash value for each guess, this 
takes time and this time is what is saved by using a hash table.
However, this comes with some compromises, the hash tables (and even rainbow 
tables) requires a lot of computational resources to produce.
They also requires great resources to use, they must preferably fit in the 
computers primary memory.
Hence the use of hash tables and rainbow tables is a trade-off between 
computational and storage resources.

Because of the space limitations of this method it can easily be countered by 
adding a salt to the hash.
This means that the rainbow table must increase too much in size to be 
feasible.
Unfortunately, some Windows hashes are not salted, so this method can be used 
on those hashes (at least in some cases).
UNIX-like systems has a longer tradition of using salts, so this method is not 
feasible on those hashes.

You will find ophcrack(1) in the package manager of most UNIX-like systems.
You can also find it on URL
\begin{center}
  \url{http://ophcrack.sourceforge.net/}.
\end{center}
You also need a few rainbow tables to be able to use the program.
You can find these on the website above.
Choose your tables carefully.

\paragraph{John the Ripper}

John the Ripper is a terminal-based program using many different ways of 
cracking passwords.
It has the possibility of brute-force attacks, dictionary attacks, and the 
possibility of using rules to modify the words in the dictionary (\eg 
\enquote{leet-speak}).
Naturally, these methods takes much longer time to use than a rainbow table, 
since all computations are done in real-time.

The program can be found in the package manager of most UNIX-like systems, or 
on URL
\begin{center}
  \url{http://www.openwall.com/john/}.
\end{center}
You are recommended to use the \enquote{Community Enhanced Version}.

To have a short summary of the possible arguments to pass to John the Ripper, 
just run the command \enquote{john} in the terminal without any arguments.
See \cref{lst:john}.
You can also read the manual page john(1).

\begin{lstlisting}[float,caption={Output from John the Ripper in the 
terminal.},label={lst:john},breaklines=false]
$ john
John the Ripper password cracker, version 1.7.8
Copyright (c) 1996-2011 by Solar Designer
Homepage: http://www.openwall.com/john/

Usage: john [OPTIONS] [PASSWORD-FILES]
--single                   "single crack" mode
--wordlist=FILE --stdin    wordlist mode, read words from FILE or stdin
--rules                    enable word mangling rules for wordlist mode
--incremental[=MODE]       "incremental" mode [using section MODE]
--external=MODE            external mode or word filter
--stdout[=LENGTH]          just output candidate passwords [cut at LENGTH]
--restore[=NAME]           restore an interrupted session [called NAME]
--session=NAME             give a new session the NAME
--status[=NAME]            print status of a session [called NAME]
--make-charset=FILE        make a charset, FILE will be overwritten
--show                     show cracked passwords
--test[=TIME]              run tests and benchmarks for TIME seconds each
--users=[-]LOGIN|UID[,..]  [do not] load this (these) user(s) only
--groups=[-]GID[,..]       load users [not] of this (these) group(s) only
--shells=[-]SHELL[,..]     load users with[out] this (these) shell(s) only
--salts=[-]COUNT           load salts with[out] at least COUNT passwords only
--format=NAME              force hash type NAME: DES/BSDI/MD5/BF/AFS/LM/crypt
--save-memory=LEVEL        enable memory saving, at LEVEL 1..3
$
\end{lstlisting}

The most interesting arguments are
\begin{itemize}
  \item --show,
  \item --wordlist,
  \item --rules, and
  \item --incremental=all.
\end{itemize}
Note that \enquote{--wordlist} and \enquote{--stdin} are separate arguments.
The first reads words from a file (provided a filename) while the latter reads 
words from standard input.
You can read more about this in the manual by typing \enquote{man 1 john} in 
the terminal.

You will find links to different wordlists to use in the following URL\@:
\begin{center}
  \url{http://sectools.org/tag/crackers/}.
\end{center}
Choose your wordlists with care.
You also have the script \enquote{pwdstream.py} (\cref{pwdstream}) to help 
generate a stream of passwords, see \enquote{./pwdstream.py -h} for details.


\subsection{Password guess generator}%
\label{pwdstream}

For this lab there is also a password guess generator.
This can be used to better control what guesses are used while cracking.
It will output a stream of passwords, one per line, on standard out, hence you 
can pipe this to John the Ripper using the \enquote{--stdin} option.

You can find its source code downloadable from the URL
\begin{center}
\url{https://github.com/OpenSecEd/passwd/releases/download/v1.1/pwdstream.py}.
\end{center}

%\lstinputlisting{pwdstream.py}


\subsection*{Acknowledgement}

This work was originally based on previous work by Rahim Rahmani and Curt-Olof 
Klasson.
It has evolved much since then, essentially only the Windows password hash is 
the same.

This work is released under the Creative Commons Attribution-ShareAlike 3.0 
Unported license.
To view a copy of this license, visit 
\url{http://creativecommons.org/licenses/by-sa/3.0/}.
You can find the original source code in URL 
\url{https://github.com/OpenSecEd/passwd/pwdguess/}.



