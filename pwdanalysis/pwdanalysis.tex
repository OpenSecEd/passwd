% $Id$
\documentclass[a4paper]{article}
\usepackage[swedish]{babel}
\usepackage[utf8]{inputenc}
\usepackage[hyphens]{url}
\usepackage{hyperref}
\usepackage{prettyref,varioref}
\usepackage{amssymb,amsmath,amsthm}
\usepackage[today,nofancy]{svninfo}
\usepackage[natbib=true,style=numeric-comp,sorting=none]{biblatex}
\addbibresource{literature}
\usepackage[prettyref,varioref]{miunmisc}

\svnInfo $Id$

\title{Grundläggande lösenordsanalys}
\author{Daniel Bosk\footnote{%
    Detta verk är tillgängliggjort under licensen Creative Commons 
    Erkännande-DelaLika 2.5 Sverige (CC BY-SA 2.5 SE).
    För att se en sammanfattning och kopia av licenstexten besök URL 
    \url{http://creativecommons.org/licenses/by-sa/2.5/se/}.
  }
}
\date{\svnId}

\newtheorem{theorem}{Sats}
\newrefformat{thm}{Sats \ref{#1}}
\newtheorem{proposition}{Proposition}
\newrefformat{prop}{Proposition \ref{#1}}
\newtheorem{lemma}{Lemma}
\newrefformat{lem}{Lemma \ref{#1}}
\newtheorem{corollary}{Korollarium}
\newrefformat{cor}{Korollarium \ref{#1}}
\theoremstyle{definition}
\newtheorem{definition}{Definition}
\newrefformat{def}{Definition \ref{#1}}
\newtheorem{example}{Exempel}
\newrefformat{ex}{Exempel \ref{#1}}
\newtheorem{exercise}{Övning}
\newrefformat{xrc}{Övning \ref{#1}}
\theoremstyle{remark}
\newtheorem{remark}{Anmärkning}
\newrefformat{remark}{Anmärkning \ref{#1}}
\renewcommand{\qedsymbol}{Q.E.D.}

\begin{document}
\maketitle
\tableofcontents

%\section{Om val}
%\noindent
%Vi stöter ofta på val av olika slag.
%I detta papper ska vi reflektera över antalet möjliga utfall som kan uppstå
%genom att olika val kombineras samman.
%
%Vi inleder med definitioner av de begrepp vi kommer att använda.
%
%\begin{definition}\label{def:Val}
%	Med \emph{val} menas att det finns \(n \geq 1\) antal distinkta alternativ
%	att välja mellan, av dessa alternativ måste ett och endast ett väljas.
%\end{definition}
%
%\begin{definition}\label{def:Utfall}
%	Ett \emph{utfall} är ett resultat av att ett val har genomförts.
%	Ett val med \(n\) alternativ får \(n\) utfall.
%\end{definition}
%
%Vi ska fortsätta med ett väldigt enkelt lemma\footnote{%
%	Det vill säga en hjälpsats.
%} om förhållandet mellan valets antal alternativ och det möjliga antalet utfall.
%Även om detta lemma anses självklart ska vi ändå inkludera ett bevis för
%det.
%För att bevisa det behöver vi dock Dirichlets lådprincip, vilken visas härnäst.
%
%\begin{theorem}[Dirichlets lådprincip]
%	Låt \(k\) objekt fördelas över \(n\) lådor.
%	Om \(k > n\), då måste någon låda innehålla fler än ett objekt.
%\end{theorem}
%
%\begin{proof}
%	Vi bevisar satsens kontrapositiva utsaga\footnote{%
%		Betrakta utsagan \emph{Om vi är i Sundsvall, då är vi i Sverige}.
%		Dess kontrapositiva utsaga är \emph{Om vi \emph{inte} är i Sverige, då
%		är vi \emph{inte} i Sundsvall}.
%		En utsagas kontrapositiva utsaga är logiskt ekvivalent med utsagan
%		själv.
%		Det vill säga om den ena är sann, då är den andra också det.
%	}, nämligen om ingen låda innehåller fler än ett objekt då måste
%	\(k \leq n\).
%
%	Vi antar att ingen av de \(n\) lådorna innehåller fler än ett objekt.
%	Låt \(m_i\) beteckna antalet objekt i låda \(i\), enligt vårt antagande har
%	vi då att \(m_i \leq 1\) för alla \(1 \leq i \leq n\).
%	Om vi summerar antalet objekt får vi \[k = \sum_{i=1}^n m_i = m_1 + m_2 +
%	\cdots + m_n \leq \underbrace{1 + 1 + \cdots + 1}_n = n.\]
%	Således är \(k \leq n\) och vi har då visat satsen.
%\end{proof}
%
%Då kan vi ta oss ann vårt första lemma.
%
%\begin{lemma}\label{lem:AntalUtfall}
%	Ett val från \(n\) alternativ har \(n\) möjliga utfall.
%\end{lemma}
%
%\begin{proof}
%	För varje alternativ behövs åtminstone ett utfall.
%	Om vi har \(n\) alternativ och antar att vi får \(n+1\) utfall,
%	då måste det enligt Dirichlets lådprincip vara något alternativ som får
%	fler än ett utfall.
%	Men om ett och samma alternativ har flera utfall, då måste dessa utfall
%	vara samma utfall.
%	Detta är en motsägelse\footnote{%
%		En utsaga \emph{Om A då B} är logiskt ekvivalent med utsagan \emph{Om A
%		och icke B då motsägelse}.
%		Vi antog att vi hade \(n\) alternativs och att detta ledde till fler än
%		\(n\) utfall.
%		Detta ledde till en motsägelse, således är det sant att 
%	} och därför måste vi ha exakt \(n\) utfall.
%\end{proof}
%
%För fullständighet inkluderas även ett exempel för att illustrera begreppen.
%
%\begin{example}\label{ex:Fikabrod}
%	Du ska välja fikabröd till eftermiddagsfikat.
%	De alternativ du har att välja mellan är en nybakad bulle, en torr kaka och
%	att inte ta något fikabröd.
%	Notera att välja \emph{ingenting} utgör ett alternativ, det går alltså inte
%	att avstå från ett val.
%
%	Valet av fikabröd har tre alternativ, det finns
%	%enligt \prettyref{lem:AntalUtfall}
%	således tre möjliga utfall.
%	Ett utfall är att vi väljer bullen, ett annat att vi väljer kakan och det
%	sista att vi väljer att inte ta något fikabröd.
%\end{example}
%
%
%
%\subsection{Att välja bland val}
%\noindent
%Då är det dags att utöka våra möjligheter att välja genom att kombinera flera
%val till ett sammansatt val.
%
%\begin{definition}
%	När ett val ska göras följt efter ett annat säger vi att vi har ett
%	\emph{sammansatt val}.
%	Ett sammansatt val kan ibland kallas för val.
%	Varje ingående val kallas för ett \emph{delval}.
%\end{definition}
%
%\begin{example}
%	Det är dags för eftermiddagsfika igen.
%	Du ska först välja om du ska dricka kaffe, te eller vatten.
%	(Att inte välja någonting är inte ett alternativ i detta val, detta är ett
%	rent teoretiskt val eftersom att rent praktiskt finns ju alltid
%	alternativet att inte fika alls.)
%	Därefter ska du välja fikabröd enligt \prettyref{ex:Fikabrod}.
%	Då har vi ett sammansatt val bestående av två delval, ett för dryck och ett
%	för fikabröd.
%\end{example}
%
%Det är nu intressant att veta hur många möjliga utfall som ett sådant val
%möjligen kan ha.
%Detta sammanfattas i följande sats.
%
%\begin{theorem}[Multiplikationsprincipen]\label{thm:Multiplikationsprincipen}
%	Ett sammansatt val av \(m\) antal delval, där delval \(i\) har \(n_i\)
%	antal alternativ,
%	har \(n_1\cdot n_2\cdots n_m\) antal möjliga utfall.
%\end{theorem}
%
%\begin{proof}
%	Låt oss börja med att titta på det sista delvalet, val \(m\).
%	Detta val har \(n_m\) alternativ och således \(n_m\) utfall.
%	%enligt \prettyref{lem:AntalUtfall}.
%	Vi går vidare till valet innan, det vill säga val \(m-1\).
%	Detta val har \(n_{m-1}\) möjliga utfall.
%	För varje utfall av detta val kan vi få \(n_m\) utfall i val \(m\).
%	Då har vi alltså tillsammans
%	\begin{equation}\label{eq:MultprincipTvaVal}
%		\sum_{k=1}^{n_{m-1}} n_m = \underbrace{n_m + n_m + \cdots
%		n_m}_{n_{m-1}} = n_{m-1}\cdot n_m.
%	\end{equation}
%	För varje utfall av delval \(m-2\) kan vi få antalet utfall från
%	\prettyref{eq:MultprincipTvaVal}.
%	Det vill säga
%	\begin{equation}
%		\sum_{k=1}^{n_{m-2}} n_{m-1}\cdot n_m = n_{m-2}\cdot n_{m-1} \cdot n_m.
%	\end{equation}
%	Vi fortsätter på detta vis tills vi når det första valet då vi får
%	\begin{equation}
%		\sum_{k=1}^{n_1} n_2\cdot n_3\cdots n_{m-2}\cdot n_{m-1}\cdot n_m
%			= n_1\cdot n_2\cdots n_m,
%	\end{equation}
%	vilket visar satsen.
%\end{proof}
%
%Från \prettyref{thm:Multiplikationsprincipen} följer direkt ett enkelt resultat
%som vi ger i detta korollarium\footnote{%
%	Det vill säga en följdsats.
%}.
%
%\begin{corollary}\label{cor:SammansattValKonstAlternativ}
%	Ett sammansatt val av \(m\) antal delval där varje delval har \(n\)
%	alternativ, har \(n^m\) antal möjliga utfall.
%\end{corollary}
%
%\begin{proof}
%	Om vi har ett sammansatt val av \(m\) antal delval, där delval
%	\(i\) har \(n_i\) alternativ.
%	Då har vi enligt \prettyref{thm:Multiplikationsprincipen} att det totala
%	utfallet är \(n_1\cdot n_2\cdots n_m\).
%	Men eftersom att alla val hade samma antal alternativ, nämligen \(n\), då
%	får vi att \(n_1\cdot n_2\cdots n_m = n\cdot n\cdots n = n^m\).
%	Följaktligen får vi \(n^m\) antal möjliga utfall då vi har \(m\) delval där
%	varje delval har \(n\) alternativ.
%\end{proof}


\section{Val av lösenord och en enkel metrik för lösen\-ords\-styrka}
\label{sec:choice}
Vi ska nu titta på hur detta kan användas för att undersöka säkerheten hos
lösenord.
Vi inleder med att definiera styrkan hos ett lösenord.
% XXX use logarithm to have a normalised password metric which is
% XXX easily comparable.
% XXX also include an aspect of time for the metric, thus we include the
% XXX ''change password every three months'' aspect of password policy.
\begin{definition}\label{def:passwdstrength}
  \emph{Styrkan} hos ett lösenord av längd \(n\) som väljs från ett alfabet 
  \(A\) är \(|A|^n,\) det vill säga antalet tecken i vårt alfabet upphöjt till 
  längden av lösenordet.
\end{definition}
Anledningen till att vi väljer \prettyref{def:passwdstrength} är för att detta 
är antalet möjliga lösenord tillika antalet gissningar som krävs, i värsta 
fall, för att gissa rätt lösenord.
Det vill säga, desto fler antal gissningar ett lösenord kräver ju längre tid 
tar det att gissa och alltså är det säkrare.
Rent statistiskt sett kommer antalet gissningar som krävs att vara hälften av 
vårt mått för lösenordsstyrkan eftersom att i medel kommer vi att behöva gå 
igenom hälften av antalet lösenord.

Det finns flera angreppssätt för att skapa lösenord, exempelvis genom att
välja en kombination av tecken; då har vi bokstäver, siffror och specialtecken 
som alfabet.
Det går också att skapa lösenord genom att slumpmässigt välja några ord som
kombineras till ett lösenord; i detta fall utgör ordlistan som vi väljer från 
vårt alfabet, exempelvis Svenska Akademiens ordlista.

Vi börjar med det första fallet, där vi skapar ett lösenord genom att kombinera
tecken.
\begin{example}\label{ex:characters}
  Om vi ska skapa ett lösenord som är fem tecken långt och får innehålla
  bokstäverna A-Z, både gemener och versaler, siffrorna 0-9 samt specialtecknen 
  ''!@.\%\&'', då kan vi se att vårt alfabet \[A = \{\text{A}, \text{B}, 
  \ldots, \text{Z}, \text{a}, \text{b}, \ldots, \text{z}, 0, 1, \ldots, 9, 
  \text{!}, \text{@}, \text{.}, \text{\%}, \text{\&}\}.\]
  Vårt alfabet innehåller således \(|A| = 26\cdot 2 + 10 + 5 = 67\) tecken.
  Styrkan hos ett sådant lösenord är följaktligen \(67^5 = 1350125107.\)
\end{example}

Nu fortsätter vi med att titta på fallet med att välja lösenord genom att
slumpmässigt välja några ord.
\begin{example}\label{ex:words}
  Vi bestämmer oss för att använda ett lösenord med fyra slumpmässigt valda
  ord från svenska språket.
  Detta ger oss ett alfabet \[A = \{\text{apa}, \text{banan}, \text{hej}, 
  \text{dator}, \ldots\},\]
  där \(A\) alltså innehåller samtliga ord i Svenska Akademiens ordlista.
  Enligt Svenska Akademien innehåller ordlistan ungefär 125000 ord \cite{SAOL}, 
  detta ger oss \(|A| \approx 125000\).
  Enligt \prettyref{def:passwdstrength} får vi då att styrkan för ett lösenord 
  av denna typ är
  \begin{equation}\label{eq:AntalUtfallOrd}
    |A|^4 \approx 125000^4 = (2^3\cdot 5^6)^4
      = 2^{12}\cdot 5^{24}
      = 2^{12}\cdot 25^{12}
      = (2\cdot 25)^{12} = 50^{12},
  \end{equation}
  det vill säga \(244140625000000000000.\)
\end{example}
Det senaste fallet kan också ses ur det första fallets perspektiv.
Om vi tittar på det sista ledet i likheten i \prettyref{eq:AntalUtfallOrd} ser 
vi direkt att detta skulle exakt motsvara ett alfabet med 50 tecken och en 
lösenordslängd av 12 tecken.


\section{Att angripa lösenord}
\label{sec:attack}
Den egentliga skillnaden mellan \prettyref{ex:characters} och 
\prettyref{ex:words} är de möjliga angreppssätten.
Exempelvis går det att angripa \prettyref{ex:words} på ett sätt som gör att 
dess styrka blir större.

För att kunna knäcka ett lösenord måste en gissning kunna testas.
Detta kan göras på flera sätt.
Exempelvis kan vi skicka en gissning till inloggningsprogrammet, exempelvis 
inloggningsgränssnittet för en webbmail.
Detta tar oftast lång tid och gör systemet i fråga varse om ett angreppsförsök.
Alternativet är att vi har tillgång till lösenordsdatabasen och kan direkt 
testa mot den.
Detta är inte ovanligt 
\citep[jmf.][]{Cubrilovic2009rhf,Oberheide2010bao,Hunt2011abs,Cluley2012twp}.
I många system lagras inte själva lösenorden utan ett hashvärde av lösenordet.
Det vi gör för att testa vår gissning är att vi beräknar hashvärdet för 
gissningen och jämför detta med det värde som finns i lösenordsdatabasen.
Notera att eftersom många användare återanvänder sina lösenord i flera system 
behöver vi inte nödvändigtvis ha lösenordsdatabasen för det system vi är 
intresserade av att angripa, det räcker med att det finns användare som har 
konton i båda systemen.

Det finns huvudsakligen tre olika former av lösenordsknäckning.
Dessa är
\begin{itemize}
  \item forcering (brute-force attack),
  \item ordlistemetoden (dictionary-attack), och
  \item i övrigt kvalificerade gissningar.
\end{itemize}
\foreignlanguage{english}{Social engineering} är egentligen inte en 
lösenordsknäckningsstrategi utan en generell teknik för att ta sig förbi 
åtkomstkontroll \citep[s. 18]{Anderson2008sea}, men eftersom att 
åtkomstkontroll vanligtvis implementeras med lösenordsskydd är den värd att 
nämna i sammanhanget.
Varför knäcka lösenordet när du kan be användaren att utföra attacken åt dig?

Forcering innebär att vi låter ett program gå igenom alla möjliga 
teckenkombinationer i vårt alfabet för att finna lösenordet i fråga.
Med denna metod är vi garanterade att finna lösenordet, men det kan dock ta 
väldigt lång tid.

Ordlistemetoden är en effektivare metod, med denna metod använder vi en 
ordlista med de vanligast förekommande lösenorden och vi låter dessa vara våra 
gissningar.
Det är vanligt att lösenordslistor skapas av publicerade hackade databaser, 
vilket är fallet i analyserna gjorda av 
\citet{Cubrilovic2009rhf,Oberheide2010bao,Hunt2011abs,Cluley2012twp}.
Detta ger färre antal gissningar och går således snabbare, men om lösenordet vi 
försöker att knäcka inte finns med i ordlistan kommer vi aldrig att kunna 
knäcka det med denna metod.

% XXX add a note on rainbow tables
%\subsection{En notis om regnbågstabeller}
%\dots

\subsection{Forcering applicerad på \prettyref{ex:words}}

Låt oss anta att medellängden av orden i Svenska Akademiens ordlista är fem 
tecken och att dessa tecken enbart är gemener från det svenska språket.
Det innebär att \prettyref{ex:words} ger oss en lösenordsstyrka på
\begin{equation}\label{eq:AntalUtfallTecken}
  29^{5\cdot 4} = 29^{20} = 176994576151109753197786640401.
\end{equation}
Med hjälp av logaritmen kan vi enkelt se vilken av dessa som är störst.
Vi har att \(\log(29^{20}) \approx 29\) medan \(\log(50^{12}) \approx 20.\)
Att konstruera lösenordet utifrån fyra slumpmässigt valda ord är alltså mycket 
starkare sett ur detta perspektiv.

Hur spelar denna representation någon roll, vad betyder skillnaden mellan
\prettyref{eq:AntalUtfallOrd} och \prettyref{eq:AntalUtfallTecken}?
Det första som bör påpekas är att i \prettyref{eq:AntalUtfallTecken} tas även
teckenkombinationer som ej är svenska ord med.
Detta eftersom att valet var att välja fyra uppsättningar av fem tecken.
Betydelsen av detta är att om vi låter en dator bara slumpa fram 20 bokstäver
(gemener), då kommer det att resultera i antalet gissningar från
\prettyref{eq:AntalUtfallTecken}.
Men det är inte ens säkert att datorn kommer att hitta rätt lösenord om vi 
råkade välja fyra långa ord som alla var minst sex bokstäver långa.
Detta är ett problem med denna uppskattning.

Om vi däremot använder Svenska Akademiens ordlista som alfabet när vi tillämpar 
forcering då kommer antalet gissningar att maximalt bli de från
\prettyref{eq:AntalUtfallOrd} och vi kommer garanterat att finna lösenordet.

Utifrån detta kan vi konstatera att denna enkla modell är för enkel för att 
säkert kunna resonera om styrkan hos olika typer av lösenord.
Vår modell här kan användas för att på ett enkelt sätt översiktligt jämföra 
styrkan hos olika typer av lösenord.
Det har däremot forskats fram mer formella modeller, vilket vi ser i nästa 
avsnitt, som kan användas för att resonera kring starka och svaga lösenord.

% XXX rewrite example from exam on effects of password policies
\subsection{Effekten av en lösenordspolicy}

Den något enkla lösenordspolicyn som kräver minst åtta tecken med gemener, 
versaler och siffror -- utan krav på något antal inom de olika kategorierna -- 
ger \( (26 + 26 + 10)^8 = 62^8 \approx 2^{48} \) antal lösenord.
Denna policy har inga krav på giltighetstid hos ett lösenord.

Universitetets lösenordspolicy kräver minst åtta tecken.
Dessa tecken ska vara minst tre gemener, tre versaler och två siffror -- 
dessutom måste dessa finnas bland de första åtta tecknen i lösenordet.
Detta ger \( 26^3 26^3 10^2 = 26^6 10^2 \approx 2^{35} \) antal möjliga 
lösenord.
Lösenordet måste dessutom bytas var tredje månad, vilket i sin tur ger risken 
för lösenordssystem där användaren baserar det nya lösenordet på det gamla.

Resultatet av detta är en reduktion av komplexiteten från \( 62^8 \) ned till 
\( 26^6 10^2 \).
Detta utgör en relativ minskning av komplexiteten med \( 1-\frac{26^6 
10^2}{62^8} = 0.99986 \), alltså 99.99 procent.
Om vi bortser från journalistformuleringen i föregående mening och tar det 
akademiska perspektivet ser vi att den första policyn är cirka \( 2^{13} = 8192 
\) gånger mer komplex.

Oavsett vilken av ovan givna policyer som används får användarna (sannolikt) 
svaga lösenord.

\begin{exercise}
  Förklara hur dessa lösenordspolicyer kan angripas.
\end{exercise}

\begin{exercise}
  Ge ett förslag på en riktigt bra lösenordspolicy.
  Glöm inte att en lösenordspolicy är meningslös utan tillhörande analys av 
  den.
\end{exercise}

\subsection{Forskning på området}

Angreppsmetoder mot och hur användare väljer lösenord och -fraser är ett aktivt 
forskningsområde.
Metoderna blir alltmer avancerade och exempelvis undersöker 
\citet{Bonneau2012lpo} hur lingvistiken påverkar valet av lösenord beståendes 
av flera ord.
\citeauthor{Bonneau2012lpo} finner att användarna inte väljer slumpmässiga ord 
utan föredrar att välja dem anpassade efter naturligt språk.
Exempelvis XKCD:s ''correct horse battery staple''\footnote{%
  URL: \protect\url{http://xkcd.com/936/}.
} föredras framför ''horse correct battery staple'' på grund av att det första 
alternativet är mer grammatiskt korrekt.

\citet{Kuo2006hso} gjorde en undersökning av hur användare skapar lösenord som 
är lätta att komma ihåg.
\citeauthor{Kuo2006hso} undersökte styrkan hos frasbaserade lösenord.
Det vill säga lösenord som skapas utifrån en mening, exempelvis Googles exempel 
''To be or not to be, that is the question''\footnote{%
  URL: \protect\url{http://www.lightbluetouchpaper.org/2011/11/08/want-to-create-a-really-strong-password-dont-ask-google/}.
} som ger lösenordet ''2bon2btitq''.
Det visade sig då i undersökningen att denna typ av lösenord är lite säkrare än 
medellösenordet, men användare väljer fortfarande lösenord som är lätta att 
gissa.

\citet{Komanduri2011opa} genomförde också en undersökning om lösenordsstyrka 
och hur lösenordspolicyer påverkar valet av lösenord.
Dessa använder Shannons entropi \cite{Shannon1948amt} som metrik för 
lösenordsstyrka.
Denna metrik är väl anpassad för denna typ av undersökning, den går dock inte 
att tillämpa utan tillgång till en större samling av valda lösenord.
De fann att den lösenordspolicy som gav starkast valda lösenord var den enkla 
policyn att lösenordet ska vara minst 16 tecken långt, inga andra krav.
Denna policy visade sig även vara den som gav lösenord som var enklast att 
komma ihåg.

Utöver de ytliga analyserna av läckta lösenordsdatabaser som gjordes av 
\citet{Cubrilovic2009rhf,Oberheide2010bao,Hunt2011abs,Cluley2012twp} har 
\citet{Bonneau2012sog} gjort en mer formell och djupgående analys.
\citeauthor{Bonneau2012sog} har tittat på lösenorden hos nästan 70 miljoner 
Yahoo!-användare och har då kunnat undersöka skillnader mellan olika 
demografiska grupper.
I artikeln utvecklas en metrik för styrkan hos lösenord, en mer formell och 
ingående än den mycket simpla metrik some ges i \prettyref{def:passwdstrength}.

\citet{Bonneau2012ghs} har skrivit sin avhandling om tillvägagångssätt för att 
gissa hemligheter valda av människor, där lösenord är en självklar sådan 
hemlighet.
I avhandlingen presenteras en matematisk modell för mänskligt val och en metrik 
för att modellera motståndskraften mot olika gissningsattacker.


% XXX write section on better metric for password strength using
% XXX Shannon entropy
%\section{En förbättrad metrik för lösenordsstyrka}
%\dots


\printbibliography
\end{document}
