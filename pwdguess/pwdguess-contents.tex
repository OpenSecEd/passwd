\title{%
  Lab: Password Cracking and Social Engineering
}
\subtitle{%
  Breaking authentication mechanisms
}
\author{%
  Daniel Bosk \and Lennart Franked
}
\institute{%
  Department of Information and Communication Systems\\
  Mid Sweden University, SE-851\,70 Sundsvall
}

\maketitle


\section{Introduction}
\label{sec:Introduktion}
The most common method of authentication used today is the use of passwords.
This assignment treats the security of passwords: how they should be chosen to 
give any form of security and how easily different types of passwords are 
broken.
In certain cases however, the strength of the password does not matter.
Why crack the password when it is easier to manipulate the holder to giving up 
the password?
Or simply doing the operation you want for you?
This falls under the category social engineering.
This assignment considers some aspects of this together with \acp{APT}.

\subsection{Aim}
\label{sec:Syfte}
The aim of this assignment is that you should reflect on the strength of 
different types of passwords in general, but specifically when used with 
different protection strategies, e.g.~with or without salts.

The intended learning outcome is that by completion you should:
\begin{itemize}
	\item \emph{comprehend and apply} research results in the area of security and 
  usable security.

\item \emph{combine} results from different areas to analyse different aspects.

\item \emph{evaluate} the security and usability of different designs.

\end{itemize}

The next section covers what you must read before you understand this 
assignment and how to do the work.
\Cref{sec:tasks} covers the work to be done, i.e.~how you should learn this.
\Cref{sec:exam} covers how it will be examined, i.e.~how you show that you have 
fulfilled the intended learning outcomes given above.


\section{Theory}
\label{sec:theory}
Before doing this laboratory assignment you should read Chap.~2 ``Usability and 
Psychology'' and Chap.~5 ``Cryptography'' in 
\citetitle{Anderson2008sea}~\cite{Anderson2008sea}.
Further, you need a basic understanding of information 
theory~\cite{Shannon1948amt} for this assignment, for this you are recommended 
to read \citetitle{Ueltschi2013se}~\cite{Ueltschi2013se}.

Now that you have the basic theory, you should start reading the main material 
of this assignment.
Start by reading the papers \citetitle{Kuo2006hso}~\cite{Kuo2006hso} and 
\citetitle{Komanduri2011opa}~\cite{Komanduri2011opa}.
You should then read the follow-up paper to the latter: 
\citetitle{Komanduri2014can}~\cite{Komanduri2014can}.
Finally, you should read \citetitle{kelley2012guess}~\cite{kelley2012guess}.

After that you should read about some recent incidents where password databases 
have leaked.
There is a list of breaches maintained by the \enquote{Have I been pwned} 
service\footnote{%
  URL: \url{https://haveibeenpwned.com/}.
}, you can search for news articles to read the details.
(And we encourage you to use this service.)

For a more in-depth treatment on password guessing, you are recommended to read 
\citetitle{Bonneau2012ghs} by \citet{Bonneau2012ghs}.
However, this is not a mandatory part of the assignment.

The final part of the theory concerns social engineering.
You should read about an incident striking the security company RSA, covered in 
\citetitle{Fisher2011rsa}~\cite{Fisher2011rsa}.



\section{Assignment}
\label{sec:tasks}
The assignment consists of two parts.
The first part concerns the security of various types of passwords, you will 
first reflect and then break some passwords.
The second part is about bypassing passwords without breaking them, it covers 
the social engineering aspects of security and you will construct a social 
engineering based scenario.

\subsection{Introductory reflection}
\label{sec:reflection}
You are now going to reflect on the strength of different types of passwords.
Use the theory to properly found your reasoning.
Start by comparing how long a password consisting of lower- and upper-case 
letters, numbers and special characters must be to have the same strength as 
a password consisting of three and four randomly chooses words, respectively.
Then continue by answering the following questions:
\begin{itemize}
  \item What happens if the randomly chosen words are not that randomly chosen, 
    what if they are rather a famous quote or similar?

  \item What is your estimate for password complexity to have a secure password 
    today?
    Where is the limit in number of guesses needed to correctly guess the 
    password?
\end{itemize}

\subsection{Cracking passwords}

On the website
\begin{center}
  \url{http://sectools.org/tag/crackers/}
\end{center}
you can find a list of programs for password cracking.
The two programs recommended for this assignment are
\ac{JtR} and Ophcrack.
However, you are free to use any program to solve this.

For a UNIX-like operating system the password hashes with corresponding salts 
are stored in the file \enquote{/etc/master.passwd} on BSD-based systems such 
as OpenBSD and FreeBSD\@.
In the case of Linux-based systems such as Ubuntu, the file used is 
\enquote{/etc/shadow}.
You need privileges (root) to read this file.

The hashes on a Windows system can be acquired by the program fgdump.
This is available from URL
\begin{center}
  \url{http://www.foofus.net/~fizzgig/fgdump/}.
\end{center}

The hashes in this assignment are already extracted from these files for your 
convenience.
You are going to find the passwords for both Windows and UNIX-like systems.
The hashes are available in Sect.~\ref{sec:hashes}.
Thus, you do not have to use any program like fgdump or unshadow(8) to extract 
them.

\subsubsection{Ophcrack}

The ophcrack(1) program uses a technique called rainbow tables.
What this means is that all password and hash-value combinations are 
precomputed and stored in a huge table.
This is called a hash table.
The rainbow table is a special case of hash table, the benefit is that it is 
smaller than a conventional hash table.
The hash table reduces the problem of cracking the password to searching this 
huge table.

The alternative approach is to compute the hash value for each guess, this 
takes time and this time is what is saved by using a hash table.
However, this comes with some compromises, the hash tables (and even rainbow 
tables) requires a lot of computational resources to produce.
They also requires great resources to use, they must preferably fit in the 
computers primary memory.
Hence the use of hash tables and rainbow tables is a trade-off between 
computational and storage resources.

Because of the space limitations of this method it can easily be countered by 
adding a salt to the hash.
This means that the rainbow table must increase too much in size to be 
feasible.
Unfortunately, some Windows hashes are not salted, so this method can be used 
on those hashes (at least in some cases).
UNIX-like systems has a longer tradition of using salts, so this method is not 
feasible on those hashes.

You will find ophcrack(1) in the package manager of most UNIX-like systems.
You can also find it on URL
\begin{center}
  \url{http://ophcrack.sourceforge.net/}.
\end{center}
You also need a few rainbow tables to be able to use the program.
You can find these on the website above.
Choose your tables carefully.

\subsubsection{John the Ripper}

\ac{JtR} is a terminal-based program using many different ways of cracking 
passwords.
It has the possibility of brute-force attacks, dictionary attacks, and the 
possibility of using rules to modify the words in the dictionary 
(e.g.~\enquote{leet-speak}).
Naturally, these methods takes much longer time to use than a rainbow table, 
since all computations are done in real-time.

The program can be found in the package manager of most UNIX-like systems, or 
on URL
\begin{center}
  \url{http://www.openwall.com/john/}.
\end{center}
You are recommended to use the \enquote{Community Enhanced Version}.

To have a short summary of the possible arguments to pass to \ac{JtR}, just run 
the command \enquote{john} in the terminal without any arguments.
See List.~\ref{lst:john}.
You can also read the manual page john(1).

\begin{lstlisting}[float,caption={Output from \ac{JtR} in the 
terminal.},label={lst:john},breaklines=false]
$ john
John the Ripper password cracker, version 1.7.8
Copyright (c) 1996-2011 by Solar Designer
Homepage: http://www.openwall.com/john/

Usage: john [OPTIONS] [PASSWORD-FILES]
--single                   "single crack" mode
--wordlist=FILE --stdin    wordlist mode, read words from FILE or stdin
--rules                    enable word mangling rules for wordlist mode
--incremental[=MODE]       "incremental" mode [using section MODE]
--external=MODE            external mode or word filter
--stdout[=LENGTH]          just output candidate passwords [cut at LENGTH]
--restore[=NAME]           restore an interrupted session [called NAME]
--session=NAME             give a new session the NAME
--status[=NAME]            print status of a session [called NAME]
--make-charset=FILE        make a charset, FILE will be overwritten
--show                     show cracked passwords
--test[=TIME]              run tests and benchmarks for TIME seconds each
--users=[-]LOGIN|UID[,..]  [do not] load this (these) user(s) only
--groups=[-]GID[,..]       load users [not] of this (these) group(s) only
--shells=[-]SHELL[,..]     load users with[out] this (these) shell(s) only
--salts=[-]COUNT           load salts with[out] at least COUNT passwords only
--format=NAME              force hash type NAME: DES/BSDI/MD5/BF/AFS/LM/crypt
--save-memory=LEVEL        enable memory saving, at LEVEL 1..3
$
\end{lstlisting}

The most interesting arguments are
\begin{itemize}
  \item --show,
  \item --wordlist,
  \item --rules, and
  \item --incremental=all.
\end{itemize}
Note that \enquote{--wordlist} and \enquote{--stdin} are separate arguments.
The first reads words from a file while the latter reads words from standard 
input.
You can read more about this in the manual by typing \enquote{man 1 john} in 
the terminal.

You will find links to different wordlists to use in the following URL\@:
\begin{center}
  \url{http://sectools.org/tag/crackers/}.
\end{center}
Choose your wordlists with care.
You also have the script \enquote{pwdstream.py} (Sect.~\ref{sec:pwdstream}) to 
help generate a stream of passwords, see \enquote{./pwdstream.py -h} for 
details.


\subsection{Social engineering and \acp{APT}}

In this part of the assignment you are going to help the University's security 
group think about social-engineering-based attack-scenarios.
Your assignment is to develop a realistic scenario for a social engineering 
attack, the purpose of which is to use for educating the University staff.
As inspiration you have the literature given in the theory section above.

Also please note, this is not an encouragement to perform this attack, it is 
strictly academic.
You should contribute your scenario by posting it in the forum in the course 
platform.


\section{Examination}
\label{sec:exam}
The assignment may be solved in groups of up to two students.
To get your work examined you should hand in a report (in PDF-format) 
containing the following:
\begin{itemize}
  \item All the passwords for the given hashes.
    You must also describe how you cracked them and how long it took you.

  \item You should provide your reflections on password strength from above.
    Then you should relate these to the cracking of the passwords above.
    How does theory and practice relate?

  \item Your social engineering based APT-scenario.
    Note that this should also be published in the forum in the course platform.
\end{itemize}


\subsubsection*{Acknowledements}

This work was originally based on previous work by Rahim Rahmani and Curt-Olof 
Klasson.
It has evolved since then and the only remains is the password hashes for 
Windows and the general idea of password cracking.

This work is released under the Creative Commons Attribution-ShareAlike 3.0 
Unported license.
To view a copy of this license, visit 
\url{http://creativecommons.org/licenses/by-sa/3.0/}.
You can find the original source code in URL 
\url{https://github.com/OpenSecEd/passwd/pwdguess/}.


\appendix
\section{The password hashes}
\label{sec:hashes}
The password hashes used in this lab are included below.
You can also find then downloadable from URL
\url{http://github.com/OpenSecEd/passwd/releases/tag/v1.1/}.

win-pwd.txt:
\lstinputlisting{win-pwd.txt}

unix-passwd.txt:
\lstinputlisting{unix-passwd.txt}


\section{Password guess generator}
\label{sec:pwdstream}
For this lab there is also a password guess generator.
This can be used with \ac{JtR} to better control what guesses are used while 
cracking.
It will output a stream of passwords, one per line, on standard out, hence you 
can pipe this to \ac{JtR} using the \enquote{--stdin} option.

You can find its source code downloadable from the URL
\url{https://github.com/OpenSecEd/passwd/releases/download/v1.1/pwdstream.py}.

%\lstinputlisting{pwdstream.py}


\printbibliography{}
