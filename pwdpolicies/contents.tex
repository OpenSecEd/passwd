\title{Seminar: Password-Composition Policies}
\subtitle{Designing a secure and usable password policy}

\author{%
  Daniel Bosk
  \and
  Lennart Franked
}
\institute{%
  Department of Information Systems and Technology\\
  Mid Sweden University, SE-851\,70 Sundsvall
}

\maketitle

\begin{abstract}
  A lot of user authentication is based on passwords.
We use password policies to aid users in selecting a secure password.
Unfortunately, research has shown that the common password-polices do not have 
the expected effect: users can still choose easy-to-guess passwords and the 
policies actually makes guessing easier.
It is thus important to \emph{scientifically} evaluate the actual effects of 
any user-authentication mechanism, otherwise our security might be at risk.
Here we will focus on exactly that.
More specifically, after this lab you should be able to
\begin{itemize}
  \item \emph{evaluate} the effective security by considering security and 
    usability.
  \item \emph{analyse} research results in usable security and \emph{apply} 
    those relevant to a given situation.
  \item \emph{design} security policies aligned with usability.
\end{itemize}

To do this, we must be familiar with several topics: 
usability~\cite[Ch.~2]{Anderson2008sea}, 
cryptography~\cite[Ch.~5]{Anderson2008sea}~\cite{BoskHighLevelCrypto},  
information theory~\cite{Ueltschi2013se} and the scientific 
method~\cite{ComputerSecurityExperiments}.
The main contents is some research papers on password security and usability:
\citetitle{GuessAgainAndAgain}~\cite{GuessAgainAndAgain},
\citetitle{OfPasswordsAndPeople}~\cite{OfPasswordsAndPeople}, 
\citetitle{CanLongPasswordsBeSecureAndUsable}~\cite{CanLongPasswordsBeSecureAndUsable} 
and
\citetitle{PasswordLifeCycle}~\cite{PasswordLifeCycle};
complemented by a paper on the usability of password managers:
\citetitle{UsabilityEvaluationOfPasswordManagers}~\cite{UsabilityEvaluationOfPasswordManagers}.

\end{abstract}


\section{Introduction}
\label{sec:intro}
User authentication is present in most systems.
There is one security mechanism which can be found almost everywhere, which is 
intended to solve this problem: passwords.
From a usability perspective, this is a bad way to do user authentication.
This will, of course, also yield security implications.

Usually, in systems using passwords, the password selection of the users are 
governed by some password-composition policy to help (or force) the users to 
select strong passwords.
Thus, how these policies are designed has a great impact on the resulting 
passwords the users choose.
This impact is not always what is expected, in fact, sometimes a password 
policy can result in weaker passwords.
There is no indication that passwords will be replaced any time soon, so if we 
must use passwords, we would better use them well.

\subsection{Aim}
\label{sec:Syfte}
The aim of this seminar is to examine password-composition policies.
We want to find out how different policies affect users' password choices and 
how we can use this knowledge for designing better policies.
During this seminar you should show that you are able to:
\begin{itemize}
	\item \emph{comprehend and apply} research results in the area of security and 
  usable security.

\item \emph{combine} results from different areas to analyse different aspects.

\item \emph{evaluate} the security and usability of different designs.

\end{itemize}

\subsection{Outline}

The next section covers what you must read before you understand this 
assignment and how to do the work.
\Cref{sec:tasks} covers the work to be done, i.e.~how you should learn this.
\Cref{sec:exam} covers how it will be examined, i.e.~how you show that you have 
fulfilled the intended learning outcomes given above.


\section{Theory}
\label{sec:theory}
% $Id$
First you must read Chap.~2 \enquote{Usability and Psychology} in 
\cite{Anderson2008sea}.
Further, you need a basic understanding of information theory 
\cite{Shannon1948amt} for this assignment, for this you are recommended to read 
\citetitle{Ueltschi2013se}~\cite{Ueltschi2013se}.

Then, to participate in this seminar you must have read the papers 
\citetitle{Komanduri2011opa}~\cite{Komanduri2011opa} and 
\citetitle{Komanduri2014can}~\cite{Komanduri2014can}.
In these papers the authors have studied how different password-composition 
policies affects users' choice of passwords.



\section{Assignment}
\label{sec:tasks}
You should read the material.
While reading, write down your thoughts.
Let these questions guide your reading:
\begin{itemize}
  \item What are the main results of the research paper?
  \item How did they conclude them?
\end{itemize}

After reading and reflection, think about the following questions:
\begin{itemize}
  \item How do these results compare to your experience of what is used in 
    practice?
  \item What is your strategy for remembering passwords?
  \item How do you react to different password policies?
  \item What would be a good password-composition policy?
    Why would that be good?
  \item How much information do you think a password policy reveals about the 
    passwords?
    Is there any way we can estimate that?
  \item Is it fine to write down passwords or not?
  \item In the situation where you have forgotten you password,
  	what kind of password recovery schemes have you encountered?
  \item What is your reaction towards the different password recovery schemes?  
  \item How should password recovery be dealt with in a secure way?
  \item What problems do you perceive with authentication of users in general?
\end{itemize}

Finally, look at the University's password-composition policy, what are the 
strengths and weaknesses of this policy?


\section{Examination}
\label{sec:exam}

During the seminar we will first discuss the papers and your reflections from 
reading them.
After that we will work in groups of 3--4 students.
We will first discuss how we can estimate how much a password policy reveals 
about the passwords, i.e.\ how the password policy can make guessing easier.
Then every group will design a password-composition policy, drawing from both 
the papers and the discussions.
Finally, every group will present their policy and analysis, then we will 
evaluate it together.

To pass this assignment you need to come prepared and actively participate in 
the seminar.


\subsubsection*{\ackname}

This work was reviewed and improved by Lennart Franked, Mid Sweden University.

This work is released under the Creative Commons Attribution-ShareAlike 3.0 
Unported license.
To view a copy of this license, visit 
\url{http://creativecommons.org/licenses/by-sa/3.0/}.
You can find the original source code in URL 
\url{https://github.com/dbosk/passwd/pwdguess/}.




\printbibliography{}
