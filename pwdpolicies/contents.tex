\title{Seminar: Password-composition policies}
\subtitle{Designing a secure and usable password policy}

\author{%
  Daniel Bosk
  \and
  Lennart Franked
}
\institute{%
  Department of Information Systems and Technology\\
  Mid Sweden University, SE-851\,70 Sundsvall
}

\maketitle

\begin{abstract}
  The part of security that perhaps most affect the users is user authentication.
The predominant mechanism to achieve this is passwords.
Thus, design decisions in this are are important for both the usability and the 
security of the system.

During this seminar you will train your ability to comprehend and apply 
research results in the area of security and usable security.
You will combine results from different areas to analyse different aspects and 
to evaluate the security and usability of different designs.

We need Chap.~2 \enquote{Usability and Psychology} of \cite{Anderson2008sea}.
Further, we need a basic understanding of information 
theory~\cite{Shannon1948amt}, for this you are recommended to read 
\citetitle{Ueltschi2013se}~\cite{Ueltschi2013se}.
Finally, we will discuss the results of 
\citetitle{Komanduri2011opa}~\cite{Komanduri2011opa} and 
\citetitle{Komanduri2014can}~\cite{Komanduri2014can}.

\end{abstract}


\section{Introduction}%
\label{sec:intro}

User authentication is present in most systems.
There is one security mechanism which can be found almost everywhere, which is 
intended to solve this problem: passwords.
From a usability perspective, passwords generally perform very poorly.
This will, of course, also yield security implications.

Usually, in systems using passwords, the password selection of the users are 
governed by some password-composition policy to help (or force) the users to 
select strong passwords.
Thus, how these policies are designed has great impact on the resulting 
passwords the users choose.
This impact is not always what is expected, in fact, sometimes a password 
policy can result in weaker passwords.

There is no indication that passwords will be replaced any time soon, so if we 
must use passwords, we would better use them well.
This is the goal of this assignment.

\subsection{Aim}%
\label{sec:Syfte}

The aim of this seminar is to evaluate password-composition policies.
We want to find out how different policies affect users' password choices and 
how we can use this knowledge for designing better policies.
During this seminar you should show that you are able to:
\begin{itemize}
	\item \emph{comprehend and apply} research results in the area of security and 
  usable security.

\item \emph{combine} results from different areas to analyse different aspects.

\item \emph{evaluate} the security and usability of different designs.

\end{itemize}

\subsection{Outline}

The next section covers what you must read before you understand this 
assignment and how to do the work.
\Cref{sec:tasks} covers the work to be done, i.e.~how you should learn this.
\Cref{sec:exam} covers how it will be examined, i.e.~how you show that you have 
fulfilled the intended learning outcomes given above.


\section{Theory}%
\label{sec:theory}

Before doing this laboratory assignment you should read Chap.~2 ``Usability and 
Psychology'' and Chap.~5 ``Cryptography'' in 
\citetitle{Anderson2008sea}~\cite{Anderson2008sea}.
Further, you need a basic understanding of information 
theory~\cite{Shannon1948amt} for this assignment, for this you are recommended 
to read \citetitle{Ueltschi2013se}~\cite{Ueltschi2013se}.

Now that you have the basic theory, you should start reading the main material 
of this assignment.
Start by reading the papers \citetitle{Kuo2006hso}~\cite{Kuo2006hso} and 
\citetitle{Komanduri2011opa}~\cite{Komanduri2011opa}.
You should then read the follow-up paper to the latter: 
\citetitle{Komanduri2014can}~\cite{Komanduri2014can}.
Finally, you should read \citetitle{kelley2012guess}~\cite{kelley2012guess}.

After that you should read about some recent incidents where password databases 
have leaked.
There is a list of breaches maintained by the \enquote{Have I been pwned} 
service\footnote{%
  URL: \url{https://haveibeenpwned.com/}.
}, you can search for news articles to read the details.
(And we encourage you to use this service.)

For a more in-depth treatment on password guessing, you are recommended to read 
\citetitle{Bonneau2012ghs} by \citet{Bonneau2012ghs}.
However, this is not a mandatory part of the assignment.

The final part of the theory concerns social engineering.
You should read about an incident striking the security company RSA, covered in 
\citetitle{Fisher2011rsa}~\cite{Fisher2011rsa}.



\section{Assignment}%
\label{sec:tasks}

You should read the material.
While reading, write down your thoughts.
Let these questions guide your reading:
\begin{itemize}
  \item What are the main results of the research paper?
  \item How did they conclude them?
    \Ie what is the research method?
\end{itemize}
After reading and reflection, think about the following questions:
\begin{itemize}
  \item How do these results compare to your experience of what is used in 
    practice?
  \item What is your strategy for remembering passwords?
  \item How do you react to different password policies?
  \item What would be a good password-composition policy?
    Why would that be good?
  \item How much information do you think a password policy reveals about the 
    passwords?
    Is there any way we can estimate that?
  \item Is it fine to write down passwords or not?
  \item In the situation where you have forgotten you password,
  	what kind of password recovery schemes have you encountered?
  \item What is your reaction towards the different password recovery schemes?  
  \item How should password recovery be dealt with in a secure way?
  \item What problems do you perceive with authentication of users in general?
  \item In which situations are passwords suitable and in which are they not?
\end{itemize}
Finally, look at the University's password-composition policy, what are the 
strengths and weaknesses of this policy?

During the seminar we will first discuss the papers and your reflections from 
reading them.
After that we will work in groups of 3--4 students.
We will first discuss how we can estimate how much a password policy reveals 
about the passwords, i.e.\ how the password policy can make guessing easier.
Then every group will design a password-composition policy, drawing from both 
the papers and the discussions.
Finally, every group will present their policy and analysis, then we will 
evaluate it together.


\section{Examination}%
\label{sec:exam}

To pass this assignment you need to come prepared and actively participate in 
the seminar.


\subsubsection*{Acknowledgements}

This work was reviewed and improved by Lennart Franked, Mid Sweden University.

This work is released under the Creative Commons Attribution-ShareAlike 3.0 
Unported license.
To view a copy of this license, visit 
\url{http://creativecommons.org/licenses/by-sa/3.0/}.
You can find the original source code in URL 
\url{https://github.com/dbosk/passwd/pwdguess/}.




\printbibliography{}
