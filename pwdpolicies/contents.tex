\title{Seminar: Password-Composition Policies}
\subtitle{Designing a secure and usable password policy}

\author{%
  Daniel Bosk
}
\institute{%
  Department of Information and Communication Systems\\
  Mid Sweden University, SE-851\,70 Sundsvall
}

\maketitle


\section{Introduction}
\label{sec:intro}
One security mechanism which can be found almost everywhere is passwords.
Usually, in these systems, the password selection of the users are governed by 
some password-composition policy.
How these policies are designed has a great impact on the resulting passwords 
the users choose.
This impact is not always what is expected, in fact, sometimes a password 
policy can result in weaker passwords.

\subsection{Aim}
\label{sec:Syfte}
The aim of this seminar is to look deeper into the world of 
password-composition policies.
We want to find out how different policies affect users' password choices.
The intended learning outcome is that after participating in this seminar you 
should be able to:
\begin{itemize}
	\item \emph{comprehend and apply} research results in the area of security and 
  usable security.

\item \emph{combine} results from different areas to analyse different aspects.

\item \emph{evaluate} the security and usability of different designs.

\end{itemize}

The next section covers what you must read before you understand this 
assignment and how to do the work.
\Cref{sec:tasks} covers the work to be done, i.e.~how you should learn this.
\Cref{sec:exam} covers how it will be examined, i.e.~how you show that you have 
fulfilled the intended learning outcomes given above.


\section{Theory}
\label{sec:theory}
Before doing this laboratory assignment you should read Chap.~2 ``Usability and 
Psychology'' and Chap.~5 ``Cryptography'' in 
\citetitle{Anderson2008sea}~\cite{Anderson2008sea}.
Further, you need a basic understanding of information 
theory~\cite{Shannon1948amt} for this assignment, for this you are recommended 
to read \citetitle{Ueltschi2013se}~\cite{Ueltschi2013se}.

Now that you have the basic theory, you should start reading the main material 
of this assignment.
Start by reading the papers \citetitle{Kuo2006hso}~\cite{Kuo2006hso} and 
\citetitle{Komanduri2011opa}~\cite{Komanduri2011opa}.
You should then read the follow-up paper to the latter: 
\citetitle{Komanduri2014can}~\cite{Komanduri2014can}.
Finally, you should read \citetitle{kelley2012guess}~\cite{kelley2012guess}.

After that you should read about some recent incidents where password databases 
have leaked.
There is a list of breaches maintained by the \enquote{Have I been pwned} 
service\footnote{%
  URL: \url{https://haveibeenpwned.com/}.
}, you can search for news articles to read the details.
(And we encourage you to use this service.)

For a more in-depth treatment on password guessing, you are recommended to read 
\citetitle{Bonneau2012ghs} by \citet{Bonneau2012ghs}.
However, this is not a mandatory part of the assignment.

The final part of the theory concerns social engineering.
You should read about an incident striking the security company RSA, covered in 
\citetitle{Fisher2011rsa}~\cite{Fisher2011rsa}.



\section{Assignment}
\label{sec:tasks}
You should read the material.
While reading, write down your thoughts about the subject matter.
Let these questions guide your reading:
\begin{itemize}
  \item What are the main results of the research paper?
  \item How did they conclude them?
\end{itemize}

During the seminar we will discuss your thoughts, the results of the research 
papers, we will also discuss password-composition policies in general.
After reading and reflection, think about the following questions:
\begin{itemize}
  \item What is your strategy for remembering passwords?
  \item How do you react to different password policies?
  \item Is it fine to write down passwords or not?
\end{itemize}

Finally, look at the University's password-composition policy, what do you 
think is the reasoning behind this?


\section{Examination}
\label{sec:exam}
To pass this assignment you need to actively participate in the seminar.


\subsubsection*{Acknowledements}

This work was reviewed and improved by Lennart Franked, Mid Sweden Uiversity.

This work is released under the Creative Commons Attribution-ShareAlike 3.0 
Unported license.
To view a copy of this license, visit 
\url{http://creativecommons.org/licenses/by-sa/3.0/}.
You can find the original source code in URL 
\url{https://github.com/dbosk/passwd/pwdguess/}.


\printbibliography{}
